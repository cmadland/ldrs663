% Options for packages loaded elsewhere
\PassOptionsToPackage{unicode}{hyperref}
\PassOptionsToPackage{hyphens}{url}
%
\documentclass[
]{book}
\usepackage{amsmath,amssymb}
\usepackage{lmodern}
\usepackage{iftex}
\ifPDFTeX
  \usepackage[T1]{fontenc}
  \usepackage[utf8]{inputenc}
  \usepackage{textcomp} % provide euro and other symbols
\else % if luatex or xetex
  \usepackage{unicode-math}
  \defaultfontfeatures{Scale=MatchLowercase}
  \defaultfontfeatures[\rmfamily]{Ligatures=TeX,Scale=1}
\fi
% Use upquote if available, for straight quotes in verbatim environments
\IfFileExists{upquote.sty}{\usepackage{upquote}}{}
\IfFileExists{microtype.sty}{% use microtype if available
  \usepackage[]{microtype}
  \UseMicrotypeSet[protrusion]{basicmath} % disable protrusion for tt fonts
}{}
\makeatletter
\@ifundefined{KOMAClassName}{% if non-KOMA class
  \IfFileExists{parskip.sty}{%
    \usepackage{parskip}
  }{% else
    \setlength{\parindent}{0pt}
    \setlength{\parskip}{6pt plus 2pt minus 1pt}}
}{% if KOMA class
  \KOMAoptions{parskip=half}}
\makeatother
\usepackage{xcolor}
\IfFileExists{xurl.sty}{\usepackage{xurl}}{} % add URL line breaks if available
\IfFileExists{bookmark.sty}{\usepackage{bookmark}}{\usepackage{hyperref}}
\hypersetup{
  pdftitle={Leadership 663},
  pdfauthor={Mark Halvorson, Colin Madland, Scott Macklin},
  hidelinks,
  pdfcreator={LaTeX via pandoc}}
\urlstyle{same} % disable monospaced font for URLs
\usepackage{longtable,booktabs,array}
\usepackage{calc} % for calculating minipage widths
% Correct order of tables after \paragraph or \subparagraph
\usepackage{etoolbox}
\makeatletter
\patchcmd\longtable{\par}{\if@noskipsec\mbox{}\fi\par}{}{}
\makeatother
% Allow footnotes in longtable head/foot
\IfFileExists{footnotehyper.sty}{\usepackage{footnotehyper}}{\usepackage{footnote}}
\makesavenoteenv{longtable}
\usepackage{graphicx}
\makeatletter
\def\maxwidth{\ifdim\Gin@nat@width>\linewidth\linewidth\else\Gin@nat@width\fi}
\def\maxheight{\ifdim\Gin@nat@height>\textheight\textheight\else\Gin@nat@height\fi}
\makeatother
% Scale images if necessary, so that they will not overflow the page
% margins by default, and it is still possible to overwrite the defaults
% using explicit options in \includegraphics[width, height, ...]{}
\setkeys{Gin}{width=\maxwidth,height=\maxheight,keepaspectratio}
% Set default figure placement to htbp
\makeatletter
\def\fps@figure{htbp}
\makeatother
\setlength{\emergencystretch}{3em} % prevent overfull lines
\providecommand{\tightlist}{%
  \setlength{\itemsep}{0pt}\setlength{\parskip}{0pt}}
\setcounter{secnumdepth}{5}
\usepackage{booktabs}
\usepackage{amsthm}
\makeatletter
\def\thm@space@setup{%
  \thm@preskip=8pt plus 2pt minus 4pt
  \thm@postskip=\thm@preskip
}
\makeatother
\ifLuaTeX
  \usepackage{selnolig}  % disable illegal ligatures
\fi
\usepackage[]{natbib}
\bibliographystyle{apalike}

\title{Leadership 663}
\author{Mark Halvorson, Colin Madland, Scott Macklin}
\date{2022-04-21}

\begin{document}
\maketitle

{
\setcounter{tocdepth}{1}
\tableofcontents
}
\hypertarget{course-description}{%
\chapter*{Course Description}\label{course-description}}
\addcontentsline{toc}{chapter}{Course Description}

Examines the theoretical foundations and professional practices of coaching learners in blended-learning environments with an emphasis on facilitating transformational learning experiences. The intersection of adult learning, educational technology, and international education thought is investigated in relation to the development of effective strategies for coaching learners within the emerging context of technologically distributed global higher education. Projects develop digital literacy skills, including the use of communication, collaboration and publishing tools; and media literacy, including knowledge of copyright, open licensing, and digital citizenship.

\hypertarget{welcome}{%
\chapter*{Welcome}\label{welcome}}
\addcontentsline{toc}{chapter}{Welcome}

Welcome to LDRS 663: Effective Coaching for Transformational Learning in Blended Learning Environments!

\hypertarget{program-learning-outcomes}{%
\subsection*{Program Learning Outcomes}\label{program-learning-outcomes}}
\addcontentsline{toc}{subsection}{Program Learning Outcomes}

\begin{itemize}
\tightlist
\item
  Demonstrate effective facilitation and coaching communication skills (eg. active listening, developing rapport, providing feedback)\\
\item
  Identify a variety of facilitation/coaching methods and techniques.
\end{itemize}

\hypertarget{course-learning-outcomes}{%
\subsection*{Course Learning Outcomes}\label{course-learning-outcomes}}
\addcontentsline{toc}{subsection}{Course Learning Outcomes}

On successful completion of this course, students should be able to:

\begin{itemize}
\tightlist
\item
  analyze the characteristics of the coaching role within theoretical models of blended teaching and learning;
\item
  demonstrate the ability to model metacognitive strategies for self-regulated learning;
\item
  apply intercultural competencies in coaching learners in transformational blended learning environments;
\item
  evaluate the quality of feedback in light of evidence-based research
\item
  evaluate interactions in a learning environment and develop strategies for high quality educative interactions;
\item
  Design cognitive and social activities to meet learning outcomes.
\item
  apply multi-modal communication and collaboration tools effectively to support learning in a higher education context.
\item
  apply information and media literacies to research, produce, analyse and present information online.
\end{itemize}

\hypertarget{resources}{%
\subsection*{Resources}\label{resources}}
\addcontentsline{toc}{subsection}{Resources}

Please note that you are not required to purchase any of the folllowing resources. They are freely available on the web or accessible through the library.

\begin{enumerate}
\def\labelenumi{\arabic{enumi}.}
\tightlist
\item
  Biggs, J., \& Tang, C. (2011). Teaching for quality learning at university: What the student does (4th ed.). New York: Society for Research into Higher Education \& Open University Press. Available as eBook through TWU Library.\\
\item
  Committee on How People Learn II: The Science and Practice of Learning, Board on Behavioral, Cognitive, and Sensory Sciences, Board on Science Education, Division of Behavioral and Social Sciences and Education, \& National Academies of Sciences, Engineering, and Medicine. (2018). How People Learn II: Learners, Contexts, and Cultures. National Academies Press. \href{https://doi.org/10.17226/24783}{Link}
\item
  Vaughan, N., Cleveland-Innes, M., \& Garrison, D. (2013). Teaching in blended learning environments: Creating and sustaining communities of inquiry. Athabasca: AU Press. Retrieved from \href{http://www.aupress.ca/index.php/books/120229}{Link}\\
\item
  Bates, A. W. (2019). Teaching in a Digital Age -- Second Edition. Tony Bates Associates Ltd.~\href{https://pressbooks.bccampus.ca/teachinginadigitalagev2/}{Link}
\item
  Campbell, G. (2009). A Personal Cyberinfrastructure. EDUCAUSE Review, 44(5), 58-59. Retrieved from \url{https://er.educause.edu/articles/2009/9/a-personal-cyberinfrastructure}
\end{enumerate}

\begin{caution}
It will be assumed that you have read, understand, and agree to the
information provided on the
\href{https://www.twu.ca/student-handbook/university-policies/academic-misconduct/procedures-dealing-acts-academic-0}{`Academic
Dishonesty Policy' page}. If you have any questions at all please
contact your instructor.
\end{caution}

\hypertarget{graduate-level-writing-standards}{%
\section*{Graduate Level Writing Standards}\label{graduate-level-writing-standards}}
\addcontentsline{toc}{section}{Graduate Level Writing Standards}

For students in 663, graduate level writing standards following APA 7 are expected. Please consult the \href{https://owl.purdue.edu/owl/research_and_citation/apa_style/apa_style_introduction.html}{OWL Purdue website} for guidance and seek assistance from the TWU Writing Center and writing coaches as needed. Assignments have rubrics that attribute some marks to APA formatting and cannot be graded as fully meeting expectations if there are APA errors. That said, your conceptual understanding remains of primary importance. It is your responsibility to ensure polished work to the highest standard of which you are capable. This demands meticulous attention to detail, which will become more `natural' with practice. Please seek any necessary clarification from your instructor.

\hypertarget{assessments}{%
\chapter*{Assessments}\label{assessments}}
\addcontentsline{toc}{chapter}{Assessments}

\begin{longtable}[]{@{}
  >{\raggedright\arraybackslash}p{(\columnwidth - 4\tabcolsep) * \real{0.2308}}
  >{\centering\arraybackslash}p{(\columnwidth - 4\tabcolsep) * \real{0.3846}}
  >{\centering\arraybackslash}p{(\columnwidth - 4\tabcolsep) * \real{0.3846}}@{}}
\toprule
\begin{minipage}[b]{\linewidth}\raggedright
Assessment Task
\end{minipage} & \begin{minipage}[b]{\linewidth}\centering
Due Date
\end{minipage} & \begin{minipage}[b]{\linewidth}\centering
\% of Final Grade
\end{minipage} \\
\midrule
\endhead
Learning Reflection Blogs & Ongoing, weekly & Required but not assessed \\
Facilitated Curriculum Analysis & April 22 (completed in class) & 10\% \\
Small Group Facilitation & April 29 & 10\% \\
Peer Coaching Session & May 6 & 15\% \\
Showcase Post (choose one of your learning reflection blog posts to expand and elaborate) & May 13 & 25\% \\
Facilitation Resource Project & May 13 & 40\% \\
\bottomrule
\end{longtable}

\hypertarget{learning-pods}{%
\section*{Learning Pods}\label{learning-pods}}
\addcontentsline{toc}{section}{Learning Pods}

At the beginning of the course everyone will be placed into small groups called learning pods. these groups, of about four people, will be the place where you will practice learning facilitation and coaching principles and skills that you are learning in this course. The pods will also form the working group for doing a curriculum analysis and developing a facilitated learning resource.

\hypertarget{learning-reflection-blogs-25}{%
\section*{Learning Reflection Blogs (25\%)}\label{learning-reflection-blogs-25}}
\addcontentsline{toc}{section}{Learning Reflection Blogs (25\%)}

Throughout this course, you will be invited to write five ``working'' posts about what you are learning in this course. These posts will be published on your own WordPress blog which you will create in Unit 1 (if you don't already have one). You should consider your posts as a place for you to try out new ideas, to test your assumptions, and to share what you are learning with your community. At the end of the course you will produce a \emph{Showcase Post}, which will represent your best work. The showcase post will be the only graded post; however, your final grade will also consider how your ideas developed over the process of you writing five working draft posts.

Each of your draft posts should be 400-500 words.

\hypertarget{post-1}{%
\subsection*{Post 1}\label{post-1}}
\addcontentsline{toc}{subsection}{Post 1}

\hypertarget{due-at-the-end-of-unit-1}{%
\subsubsection*{Due at the end of Unit 1}\label{due-at-the-end-of-unit-1}}
\addcontentsline{toc}{subsubsection}{Due at the end of Unit 1}

\hypertarget{topic}{%
\subsubsection*{Topic}\label{topic}}
\addcontentsline{toc}{subsubsection}{Topic}

\begin{itemize}
\tightlist
\item
  In a new post, \href{https://ma-lead.github.io/ldrs663/building-the-web.html\#learning-activity}{complete the Visitors and Residents in online spaces Learning Activity}. What do you notice about your map? What do you wonder?
\end{itemize}

To submit all of your posts for the course, create a new post on your own WordPress blog and use the category \texttt{ldrs663}.

\hypertarget{post-2}{%
\subsection*{Post 2}\label{post-2}}
\addcontentsline{toc}{subsection}{Post 2}

\hypertarget{due-at-the-end-of-unit-2}{%
\subsubsection*{Due at the end of Unit 2}\label{due-at-the-end-of-unit-2}}
\addcontentsline{toc}{subsubsection}{Due at the end of Unit 2}

\hypertarget{read-and-discuss}{%
\subsubsection*{Read and Discuss}\label{read-and-discuss}}
\addcontentsline{toc}{subsubsection}{Read and Discuss}

\textbf{Review} \href{https://www.irrodl.org/index.php/irrodl/article/view/149/230}{Getting the mix right again: An updated and theoretical rationale for interaction} by Terry Anderson.\\
\textbf{Read} \href{https://link-springer-com.ezproxy.student.twu.ca/article/10.1007/s12528-011-9049-4}{Interaction and the online distance classroom: Do instructional methods effect the quality of interaction?} by Heather Kanuka.

Then, post a reponse on your blog defending or criticizing Anderson's Interaction Equivalency Theorem. Ensure that you defend or criticize the idea, not the person, and include something that you have learned about interaction from somewhere other than the assigned readings.

If your birthday is between January 1 and June 30, \textbf{defend} Anderson's Interaction Equivalency Theorem.

If your birthday is between July 1 and December 31, \textbf{criticize} Anderson's Interaction Equivalency Theorem

Feel free to respond to arguments presented by your colleagues for or against the theorem.

To submit this discussion post, create a new post on your own WordPress blog and use the category `ldrs663'.

\hypertarget{post-3}{%
\subsection*{Post 3}\label{post-3}}
\addcontentsline{toc}{subsection}{Post 3}

\hypertarget{due-at-the-end-of-unit-3}{%
\subsubsection*{Due at the end of Unit 3}\label{due-at-the-end-of-unit-3}}
\addcontentsline{toc}{subsubsection}{Due at the end of Unit 3}

Choose ONE of the Learning Activities in Unit 3 and respond to one or more of the prompts, or follow your own questions and thinking about the topic.

To submit this discussion post, create a new post on your own WordPress blog and use the category `ldrs663'.

\hypertarget{post-4}{%
\subsection*{Post 4}\label{post-4}}
\addcontentsline{toc}{subsection}{Post 4}

\hypertarget{due-at-the-end-of-unit-4}{%
\subsubsection*{Due at the end of Unit 4}\label{due-at-the-end-of-unit-4}}
\addcontentsline{toc}{subsubsection}{Due at the end of Unit 4}

\hypertarget{topic-1}{%
\subsubsection*{Topic}\label{topic-1}}
\addcontentsline{toc}{subsubsection}{Topic}

In your Discussion Post for this unit, you are being asked to select one core coaching competencies identified in this unit and reflect on how you might apply it in an educational setting. You can use the following questions to guide your writing:

\begin{itemize}
\tightlist
\item
  How would you define the coaching competency?\\
\item
  Why is the competency important?\\
\item
  What set of integrated knowledge, skills, aptitudes and attributes help define, in more detail, how to successfully perform the job to be done?
\end{itemize}

To submit this post, create a new post on your own WordPress blog and use the category `ldrs663'.

\hypertarget{post-5}{%
\subsection*{Post 5}\label{post-5}}
\addcontentsline{toc}{subsection}{Post 5}

\hypertarget{due-at-the-end-of-unit-5}{%
\subsubsection*{Due at the end of Unit 5}\label{due-at-the-end-of-unit-5}}
\addcontentsline{toc}{subsubsection}{Due at the end of Unit 5}

\hypertarget{topic-2}{%
\paragraph*{Topic}\label{topic-2}}
\addcontentsline{toc}{paragraph}{Topic}

Throughout this unit we have explore the idea of the educational experience. Your task for the Unit 5 Blog Post, is to reflect on recent trends in higher, and other forms, of adult education in terms of the multitude of new ways institutions are offering access educational experiences. You can use the following questions to guide your writing:

\begin{itemize}
\tightlist
\item
  How can educational institutions give learners more control over their learning experiences?\\
\item
  What benefits and challenges does learner-centred access to education introduce?\\
\item
  Is the recent move towards multi-access education shifting the site of education back to an emphasis on study and away from the focus on instruction that dominated the modern era?\\
\item
  How might this shift change the educator's role and responsibilities?\\
\item
  How might this shift change the learner's role and responsibilities change?\\
\item
  How can institutions ensure quality and transformational learning outcomes?
\end{itemize}

\hypertarget{showcase-post}{%
\subsection*{Showcase Post}\label{showcase-post}}
\addcontentsline{toc}{subsection}{Showcase Post}

\hypertarget{due-at-the-end-of-the-course}{%
\subsubsection*{Due at the end of the course}\label{due-at-the-end-of-the-course}}
\addcontentsline{toc}{subsubsection}{Due at the end of the course}

\hypertarget{topic-3}{%
\paragraph*{Topic}\label{topic-3}}
\addcontentsline{toc}{paragraph}{Topic}

Choose one of the previous 5 posts that you would like to showcase as your best work. Take some time to polish and expand your post (aim for 600-700 words). Ways to expand your post might include:
- finding more published research about the topic to integrate into your post;\\
- writing about how your views have changed on the topic during the course;
- writing a counter-argument refuting your previous post.

Please include citations (links) and a reference list at the end of your post.

To submit this discussion post, create a new post on your own WordPress blog and use the categories `ldrs663' and `showcase'.

\hypertarget{facilitated-curriculum-analysis-10}{%
\section*{Facilitated Curriculum Analysis (10\%)}\label{facilitated-curriculum-analysis-10}}
\addcontentsline{toc}{section}{Facilitated Curriculum Analysis (10\%)}

Working together as a learning pod, examine a curricular resource that is, or could be, used with a facilitated learning approach. Emphasis will be placed on courses in higher education, including a selection of TWU's library of FAR courses (specifically designed to be delivered in a facilitated learning approach). However, a range of other formats are open for investigation, including (but not limited to) community-based programs, professional certification programs, corporate workshops or training programs, and masterclasses. Your pod will be invited to select a specific course of study and asked to assess the curriculum from a facilitation and coaching perspective.

\hypertarget{small-group-facilitation-10}{%
\section*{Small Group Facilitation (10\%)}\label{small-group-facilitation-10}}
\addcontentsline{toc}{section}{Small Group Facilitation (10\%)}

\begin{itemize}
\tightlist
\item
  Working in a learning pod you will each choose, prepare, and facilitate one 15 to 20 minute long small group discussion that is based on one of the units/topics from LDRS 663. You will record a video of your session (likely in Zoom).\\
\item
  Following the facilitation, your group will engage in a 10-15 minute debrief about your facilitation. You may choose to structure the debrief following the \textbf{\emph{What, So What, Now What? W³}} from the Learning Activity in Unit 3, Topic 6.\\
\item
  Finally, you will write a 300-500 word critical reflection on your actions as the learning facilitator, including the debrief.
\end{itemize}

\textbf{Your reflection should be submitted as a post on your blog with the category \texttt{ldrs663}.}

In your facilitation session, demonstrate and reflect on the following:

\begin{itemize}
\tightlist
\item
  First, preparing for the session (as evidenced by your composed materials to support learning in your session).\\
\item
  Second, creating a supportive environment (referring to participants by name, clarifying expectations, responding in an affirming way to others, and acting in and communicating in a respectful and supportive manner).\\
\item
  Third, managing the learning process (beginning and ending the session on time, fostering participation by all learners in the group, keeping the learning group on track, facilitating interaction within the learning group, and summarizing learning, calmly and creatively adapting to unexpected events).\\
\item
  Fourth, fostering learning (showing interest and enthusiasm, spending more time asking than telling, posing open-ended questions, waiting for learners to respond, seeking clarification, activating learners' prior knowledge effectively, using appropriate forms of engagement to stimulate learner involvement).
\end{itemize}

\hypertarget{peer-coaching-session-15}{%
\section*{Peer Coaching Session (15\%)}\label{peer-coaching-session-15}}
\addcontentsline{toc}{section}{Peer Coaching Session (15\%)}

Working with another student (in your pod), you will each coach each other through the process of writing your final ``showcase'' post as part of your learning reflection blog assignment. You will record a video of your session and write critical reflection on your actions as the learning coach.

\hypertarget{facilitation-resource-project-40}{%
\section*{Facilitation Resource Project (40\%)}\label{facilitation-resource-project-40}}
\addcontentsline{toc}{section}{Facilitation Resource Project (40\%)}

Working as a learning pod, you will create a guide to serve as a resource for you and others to facilitate a particular course of study. We can't emphasize enough how important it will be for you to have analyzed, critiqued, and integrated into your practice the model of coaching and facilitation in real-world settings. As you may be facilitating learning experiences in subjects where you may not have significant domain knowledge, it will be critical for you to be able to lead students through thinking and learning processes that will lead to them discovering what they need to know from the expertly prepared course materials in order to help solve their questions.

\hypertarget{grading-standards-and-the-solo-taxonomy}{%
\section*{Grading Standards and the SOLO Taxonomy}\label{grading-standards-and-the-solo-taxonomy}}
\addcontentsline{toc}{section}{Grading Standards and the SOLO Taxonomy}

An important reality of higher education is that we need to provide a single number between 0 and 100 to the university that encapsulates the effort, successes, failures, struggles, discoveries and messiness of your work during this course. If we am going to be fair about it, we need to consider where you are starting relative to where you end up, we need to understand your individual context, and we need to be able to determine that number by researching your work.

\textbf{\emph{Assessment is research.}} You need to show us evidence that you have met the outcomes of the course in alignment with the parameters of the assignment, and you need to do so in a way that shows you can think clearly, write persuasively, and extend your learning beyond the boundaries of the course.

This is a very tall order.

One thing that you should consider is that our assessment of your work is not an assessment of you as a person. It is an assessment of what you have shown in relation to the outcomes of the course. One way that you can ensure that you are providing reflections and creating work that is of high academic quality is to~use the SOLO Taxonomy.

\hypertarget{solo-taxonomy}{%
\subsection*{SOLO Taxonomy}\label{solo-taxonomy}}
\addcontentsline{toc}{subsection}{SOLO Taxonomy}

SOLO stands for \emph{Structure of the Observed Learning Outcome} and is a gauge to help you (and me) ensure that you are writing at an appropriate level.

\begin{figure}
\centering
\includegraphics{assets/SOLO-taxonomy.png}
\caption{SOLO Taxonomy (adapted from Biggs \& Collis, 1982)}
\end{figure}

\hypertarget{pre-structural}{%
\subsubsection*{Pre-Structural}\label{pre-structural}}
\addcontentsline{toc}{subsubsection}{Pre-Structural}

A pre-structural response completely \textbf{\emph{misses the point}} of the assessment.

\hypertarget{uni-structural}{%
\subsubsection*{Uni-Structural}\label{uni-structural}}
\addcontentsline{toc}{subsubsection}{Uni-Structural}

A uni-strucutral response displays knowledge or ability in \textbf{\emph{one dimension of the construct}}.

\hypertarget{multi-structural}{%
\subsubsection*{Multi-Structural}\label{multi-structural}}
\addcontentsline{toc}{subsubsection}{Multi-Structural}

A multi-structural response displays knowledge or ability in \textbf{\emph{multiple dimensions of the construct}}, but each dimension is \textbf{\emph{disconnected}} from the others.

\hypertarget{relational}{%
\subsubsection*{Relational}\label{relational}}
\addcontentsline{toc}{subsubsection}{Relational}

A relational response displays knowledge or ability in \textbf{\emph{multiple dimensions of the construct, and how they are related to each other}}.

\hypertarget{extended-abstract}{%
\subsubsection*{Extended Abstract}\label{extended-abstract}}
\addcontentsline{toc}{subsubsection}{Extended Abstract}

An extended abstract response displays knowledge or ability in \textbf{\emph{multiple dimensions of the construct, how thy are related to each other, and how that construct can be applied to help us understand different constructs}}.

If you are providing responses at a \texttt{pre-\ or\ uni-structural} level in a university course, you are going to have a bad time. \texttt{Multi-structural} responses will lead to grades in the `C' range. At minimum, your responses should be \texttt{unambiguously\ relational} for a grade in the `B' range and \texttt{extended\ abstract} for a grade in the `A' range.

The \href{https://www.twu.ca/about/policies-guidelines/university-standard-grading-system}{TWU Grading Scale}, available on the syllabus for this course, describes A-, A, or A+ work as

\begin{quote}
\textbf{Outstanding, excellent work}; exceptional performance with strong evidence of original thinking, good organization, meticulous concern for documented evidence, and obvious capacity to analyze, synthesize, evaluate, discern, justify, and elaborate; frequent evidence of both verbal eloquence and perceptive insight in written expression; excellent problem-solving ability in scientific or mathematical contexts with virtually no computational errors; demonstrated masterful grasp of subject matter and its implications. Gives evidence of an extensive and detailed knowledge base. (Note: The A+ grade is reserved for very rare students of exceptional intellectual prowess and accomplishment, especially in lower level courses.)
\end{quote}

For a grade in the B-, B, or B+ range, here is what you need to do:

\begin{quote}
\textbf{Good, competent work}; laudable performance with evidence of some original thinking, careful organization; satisfactory critical and analytical capacity; reasonably error-free expository written expression, with clear, focused thesis and well-supported, documented, relevant arguments; good problem-solving ability, with few computational or conceptual errors in scientific subjects; reasonably good grasp of subject matter but an occasional lack of depth of discernment; evidence of reasonable familiarity with course subject matter, both concepts and key issues. Exhibits a serious, responsible engagement with the course content.
\end{quote}

We are happy to have a conversation with you if you feel your work has been unfairly assessed and you can provide a justifiable rationale based on the product of your work in relation to the requirements of the assignment and the standards outlined above and in the University Calendar.

\hypertarget{building-the-web}{%
\chapter{Building the Web}\label{building-the-web}}

\hypertarget{overview}{%
\section*{Overview}\label{overview}}
\addcontentsline{toc}{section}{Overview}

Over the first two weeks, we will introduce you to some background thinking about educational technologies. There are countless companies vying for attention in the edtech field, and they are not all educationally beneficial. In fact, some are outright harmful. We believe that it is important for you to begin to think about how your data are used when you interact with edtech tools and that it is important for you to develop some skills and competencies in building your own domain on the web. With this knowledge, you will be more prepared to spot the nefarious actors and to take control of how you present yourself online.

\hypertarget{topics}{%
\section*{Topics}\label{topics}}
\addcontentsline{toc}{section}{Topics}

\begin{enumerate}
\def\labelenumi{\arabic{enumi}.}
\tightlist
\item
  Situating yourself online
\item
  Data and Privacy
\item
  Subverting surveilance capitalism
\end{enumerate}

\hypertarget{outcomes}{%
\section*{Outcomes}\label{outcomes}}
\addcontentsline{toc}{section}{Outcomes}

\begin{itemize}
\tightlist
\item
  you will be able to articulate the importance of data and your privacy rights
\item
  you will have gained confidence in presenting your whole self online
\item
  you will be able to create and manage a single WordPress blog.
\end{itemize}

\hypertarget{resources-1}{%
\section*{Resources}\label{resources-1}}
\addcontentsline{toc}{section}{Resources}

Online resources will be provided.

\hypertarget{visitors-and-residents}{%
\section{Visitors and Residents}\label{visitors-and-residents}}

It is likely that you have encountered and may believe that there is a distinction between digital `natives' and `immigrants'.

\begin{caution}
\textbf{Note}\\
\href{https://marcprensky.com/}{Marc Prensky}, who proposed this idea,
is the one who thought it would be a good idea to refer to people as
`natives'. We recognize that this term should not be used to talk about
people.
\end{caution}

The essential argument is that \emph{kids these days} have changed in that they have this innate ability to use and learn technology because they have grown up using technology, and those of us whose formative years pre-date the advent of the internet are forever at a disadvantage compared to the \emph{kids}. You can read a bit more about the idea on Wikipedia, linked below. There is also a link in that article to Prensky's original article.

Digital native

Aside from the problematic framing of learners as kids, there are some distinct challenges with the idea of digital literacy being a fixed trait rather than a matter of comfort, familiarity, and a skill that can be practiced and learned. It is no secret that more young people are comfortable using social media apps like TikTok, Instagram, SnapChat, Weibo, WeChat, and the like, but that does not mean that those people are more able to learn technology than older people or that they have an innate ability to do so. Have you ever asked a 1st-year university student to use a spreadsheet to create a budget or a gradebook with embedded formulae? It is more likely than not, that you will encounter a distinct lack of skill in completing this task.

I'd like to introduce you to a different way to conceptualize your relationship with digital media, and that is that you may be a \emph{visitor} in some web spaces and a \emph{resident} in others. Places on the web where you might be a visitor are those places where you, quite literally, visit, but importantly, don't leave a public trace of your time there. You don't spend any time interacting with people, but rather, you take a rather utilitarian approach by visiting a site, doing a thing, and leaving.

Alternately, there are places and spaces on the web, where \emph{you} reside as a persona, where you interact, socialize, and leave traces of yourself online. For some, that may be Facebook, where you keep in touch with friends and family, or Twitter, or maybe it's a listserv you subscribed to back in the 90s, or your blog, or someone else's blog or social site. The important distinction is that these are places where you connect with other people; where you are socially \emph{present}.

At the same time, if we can imagine the visitor \textless--\textgreater{} resident continuum on a horizontal axis, there is also a personal \textless--\textgreater{} professional (or educational) continuum on a vertical axis, leading to 4 quadrants where you might situate your technology use.

The video below explains a process to help you think about where you reside on the web (7 mins).

\href{https://www.youtube.com/watch?v=sPOG3iThmRI}{Link to YouTube}

I've shared my VR Diagram below\ldots keep in mind that this diagram represents a set of tools that I have been using for a decade or more and that I have invested my career in educational technology. There is a lot here, but yours might look significantly different with only a few tools here and there. The main thing I would like to communicate with this idea of visitors and residents is for you to think about which technologies you use as a resident, and then to think about where your learners reside on the web. From there, we can begin to plan for tools we can use that afford us and our learners the opportunity to reside there.

\begin{figure}
\centering
\includegraphics{assets/u1/vr-diagram-2.png}
\caption{Visitor-Resident Diagram}
\end{figure}

It is certainly notable that I am very much a visitor in Moodle! This does not mean that I don't spend much time there, I spend a significant portion of every day working in Moodle, rather, the work that I do there leaves very little trace of my personality. You will (hopefully) see Moodle as much more of a place where you reside. But this foregrounds the question of whether Moodle is actually designed to promote residencies. Certainly the forums allow for users to project their persona into the system, as do a few of the other features, but the system itself is very heavily templated. There are profiles that can be edited, but users are limited to one very tiny image and virtually no opportunity to determine for themselves what they want to share. There is little room for customization, and every time a course ends, every single user must recreate their persona in a new course site (or five).

For many, or most, of you, Moodle is a perfectly reasonable place to reside and you are able to make learners feel at home there. We encourage that. And just like our physical homes, the quality of the community that lives there isn't determined by the features of the house itself, but by the people who share the space and how they structure their time and interactions.

If you don't already, I encourage you to subscribe to this excellent podcast called \emph{Teaching in Higher Ed} by \href{https://twitter.com/bonni208}{Bonni Stachowiak}, or, just take 47 minutes to listen to this episode in which Bonni interviews Dave White about the idea of visitors and residents.

Digital Visitors and Residents, with David White - Teaching in Higher Ed

One of the people I look up to as an educator published a blog post which I believe provides a fitting summary of this particular unit.

Technology is not Pedagogy

\hypertarget{learning-activity}{%
\subsection*{Learning Activity}\label{learning-activity}}
\addcontentsline{toc}{subsection}{Learning Activity}

\begin{wp}
\textbf{Visitor and Resident Diagram}

I hope this activity will help you think about how the tools we use
shape and sometimes determine the nature of our interactions with
learners. Do the tools you use fall on the visitor or the resident end
of your continuum? What about the learners in your classes?

✔️ \textbf{Read}
\href{https://firstmonday.org/ojs/index.php/fm/article/view/3171}{Visitors
and Residents: A new typology for online engagement}\\
✔️ Complete your own \textbf{\emph{Visitor/Resident}} map and share and
discuss it with your Learning Pod. You can use the
\href{https://experimental.worldcat.org/vandrmapping/editMap}{tool that
is provided here (DON'T forget to screenshot it, or you'll lose it!)},
or use a different tool like \href{https://canva.com}{Canva.com}.\\
✔️ Share your Visitor-Resident Diagram in a new post on your WordPress
blog. Include a short reflection on what your diagram tells you about
your web presence. Feel free to interact with others' diagrams!
\end{wp}

\hypertarget{fippa-privacy-and-consent-resources}{%
\section{FIPPA, Privacy, and Consent Resources}\label{fippa-privacy-and-consent-resources}}

It is second nature to most to take selfies and share them on Instagram, Snapchat, etc., but once you move into the role of an educator in either a public or private context, you must adhere to the laws set out by the \href{https://www.oipc.bc.ca/}{B.C. Office of Information and Privacy Commissioner}. Their office has put together guidelines for both public bodies and private bodies. TWU, as a private body is not held to the same standard as public bodies, but we should strive to meet the same standard. The guidelines for public bodies to better understand what the rules are is linked below and how to get consent is detailed on page 4 of the \href{https://esquimalt.public.sd61.bc.ca/wp-content/uploads/sites/34/2013/09/OIPC-Cloud-Computing-Guidelines-for-Public-Bodies.pdf}{BC Cloud Computing Guidelines (PDF)} and you can review the \href{http://www.bclaws.ca/Recon/document/ID/freeside/96165_00}{Freedom of Information and Protection of Privacy Act here.}. \href{https://www.twu.ca/about/university-privacy-policy}{TWU also has a privacy policy, available here.}

Each public body will have their own process (which may range from not allowing tools to pressure to integrate networked learning tools from outside of Canada), so it is important to understand your own setting and the law. You may find some administrators or staff breaking these rules or not aware of them. It is important for you to enter your field and uphold the law, regardless of the culture you enter. This does not mean that you do not engage online or outside of Canada. It means that if/when you do so, that you understand the steps, which are not much more complex than the consent you would get normally for going ``on the Internet,'' as is described in most settings, but you must name the date consent is effective and, if applicable, the date it expires. It is important that you work with your school district on the consent process. You can see an example of how K-12 school districts are addressing access to cloud tools outside of Canada \href{http://www.sd43.bc.ca/Resources/DigitalCitizenship/Pages/CloudTools.aspx}{here (Coquitlam)} and \href{https://www.sd61.bc.ca/programs/digital-learning/sd61-gafe/privacy-and-personal-information/}{here} plus \href{https://techforlearning.sd61.bc.ca/privacy/consent-process/}{here (Victoria)}. You must also name each tool individually. It cannot be ``blogging.'' You must name WordPress.com or Blogger, etc. If you use Flipgrid, you must name Flipgrid. Consent must also be informed, so effort must be taken to ensure that those signing consent understand the implications -- that their data may leave Canada, how it may be harvested, and to know about the U.S. Patriot Act. One archived resource by the Canadian Treasury Board provides significant detailed information about the \href{https://www.tbs-sct.gc.ca/pubs_pol/gospubs/TBM_128/usapa/faq-eng.asp}{Patriot Act here}. It is helpful to also review \href{https://www2.gov.bc.ca/assets/gov/education/kindergarten-to-grade-12/teach/teaching-tools/digital-literacy-framework.pdf}{section 4(b) of the B.C. Digital Literacy Framework which is applicable K12 contexts} but helpful for others.

Additional resources can be found here:

\href{https://digitaltattoo.ubc.ca/quizzes/privacy-and-surveillance}{Privacy, Ethics \& Security in Digital Spaces Developing Awareness of Privacy}

\href{https://www2.gov.bc.ca/gov/content/governments/services-for-government/information-management-technology/information-security/information-security-awareness}{Information Security Awareness} by the BC Government

\href{http://mediasmarts.ca/}{MediaSmarts: Canada's Centre for Digital and Media Literacy}

\hypertarget{fippa-privacy-and-consent-competencies}{%
\subsection*{FIPPA, Privacy, and Consent Competencies}\label{fippa-privacy-and-consent-competencies}}
\addcontentsline{toc}{subsection}{FIPPA, Privacy, and Consent Competencies}

Learners should ensure that they:\\
- Are aware of the OIPC, FIPPA, and the Cloud Computing Guidelines and follow them\\
- Understand what constitutes personal information\\
- Understand that privacy online is a personal choice and must be respected\\
- Understand that when you assume an ``educator'' hat, you have a duty for those under your care, their parents and families, and your colleagues with regard to their privacy and protection of personal information\\
- Are aware that the Canadian federal government states that the chances are remote that the US Patriot Act will access personal information of Canadians, but recognizes that it is our responsibility to protect privacy preferences and to ensure that consent obtained is informed consent. Some families may be involved with restraining orders and need to be private for their safety, but the reasons for privacy may be preference. Either way, it is not our business as to the reasons for privacy preferences, but it is our responsibility to uphold preferences.\\
- Understand how media moves through networks into US cloud-based services (e.g., back-ups on iTunes, syncing with Dropbox, messages with personal information is sent on Gmail, Google Docs, blog RSS subscriptions, etc.)\\
- Understand that these acts do not prohibit participation in networked tools outside of Canada and many public bodies are in need of staff and leaders who model networked literacy and positive citizenship online for their community\\
- Understand what appropriate consent looks like for public bodies and is aware of what alternative steps are to support learners when consent is not obtained.

\hypertarget{wordpress-and-privacy}{%
\section{WordPress and Privacy}\label{wordpress-and-privacy}}

Now that you have thought about privacy in our digital world, it is time for you to make some decisions about how you would like to `be' online.

You are invited to document your learning in WordPress. This means that your work would be posted online on a public site. Keep in mind, though, that you are NOT required to post your work publicly. The steps below can help you decide how comfortable you are with sharing publicly.

Please review all 5 steps below to decide on your approach.

\hypertarget{decide-if-you-are-comfortable-posting-your-work-online.}{%
\subsection*{Decide if you are comfortable posting your work online.}\label{decide-if-you-are-comfortable-posting-your-work-online.}}
\addcontentsline{toc}{subsection}{Decide if you are comfortable posting your work online.}

If not, you can document your learning offline (with technology) by changing the privacy settings on your blog or using Word documents and offline video. We would ask learners to consider using an online blogging tool with no identification/using a pseudonym, so as to develop network literacy, which is important in supporting learners, who are growing up in networked environments, but the preferences of learners will be respected and supported.

If you are comfortable being online, then proceed to step 2.

\hypertarget{would-you-like-to-use-your-real-name-or-use-a-pseudonym}{%
\subsection*{Would you like to use your real name or use a pseudonym?}\label{would-you-like-to-use-your-real-name-or-use-a-pseudonym}}
\addcontentsline{toc}{subsection}{Would you like to use your real name or use a pseudonym?}

You can claim your name online and own your presence by using your full name. With increasing catfishing and identity theft online, it can be helpful to have a presence that may compete with any fake profiles of you that are out there or to have a more dominant presence so posts or pictures of you by others may get drowned out. That said, you may wish to create an identity without your personal information (e.g., West coast teacher). The choice is yours.

With that decision made, proceed to step 3.

\hypertarget{decide-if-you-would-like-your-blog-to-be-hosted-outside-of-canada-or-inside-of-canada.}{%
\subsection*{Decide if you would like your blog to be hosted outside of Canada or inside of Canada.}\label{decide-if-you-would-like-your-blog-to-be-hosted-outside-of-canada-or-inside-of-canada.}}
\addcontentsline{toc}{subsection}{Decide if you would like your blog to be hosted outside of Canada or inside of Canada.}

We strongly recommend that you create a blog at create.twu.ca which is built specifically for students and faculty at TWU, is hosted within Canada, and is completely free for you to use. We also have created a template for you there, which will make getting started easier. You will not lose access to your site at create.twu.ca after you finish at TWU, but you are free to export it and publish it on your own space and on your own domain (e.g., \url{http://yourname.ca} or \url{http://westcoastteacher.ca}) with a web hosting company for a reasonable annual fee. Some of these companies host outside of Canada (e.g., Dreamhost), while others host within Canada (e.g., Canadian Web Hosting).

Be sure you review the resources under the privacy tutorial on this site or talk to your instructor about the implications of your options. You should also review the resources at the BC Office of the Information and Privacy Commissioner along with the Cloud Computing Guidelines, which outline how to get consent.

With that decision made, proceed to step 4.

\hypertarget{you-also-have-to-decide-if-you-want-to-make-your-blog-public-or-private.}{%
\subsection*{You also have to decide if you want to make your blog public or private.}\label{you-also-have-to-decide-if-you-want-to-make-your-blog-public-or-private.}}
\addcontentsline{toc}{subsection}{You also have to decide if you want to make your blog public or private.}

You can set an entire blog to be private or simply selected posts can be set to private. You can set a password or invite people to gain access. We have provided instructions for adjusting your privacy settings on the tutorial page for opened.ca.

\begin{protip}
\textbf{✨ ProTip}

If you choose to make your blog private, we will not be able to
syndicate your posts to the course site. You will still be able to
participate in the course, but please contact your instructor if you
choose to make your site only visible to registered users of the network
or your site.
\end{protip}

And last, but not least\ldots{}

\hypertarget{finally-you-have-to-think-about-where-you-and-your-content-will-end-up.}{%
\subsection*{Finally, you have to think about where you and your content will end up.}\label{finally-you-have-to-think-about-where-you-and-your-content-will-end-up.}}
\addcontentsline{toc}{subsection}{Finally, you have to think about where you and your content will end up.}

The wonderful thing about WordPress is that you can import that exported file into another WordPress instance (it sounds hard, but it isn't and we'll show you) or if you want to later set up your own domain and with your own WordPress installation. You may also import it into WordPress.com, but be aware that if you made posts with personal information knowing your site was hosted in Canada at the time and simply contained regular consent, without the specific consent for hosting outside of Canada, which requires you to name each tool, etc., you might not have consent to switch to WordPress.com. We often advise learners to post as if they will be on the cloud outside of Canada. To be honest, if you have a public blog, your friends and colleagues may be using U.S. cloud-hosted tools like Feedly to curate and read your blog posts or they may repost/quote your content on their U.S. blog. There are many educators who use U.S. software in their teaching and to support their learners. Just be sure to review how to get consent as per page four of the BC OIPC Cloud Computing Guidelines linked here.

\hypertarget{creating-a-blog}{%
\subsection*{Creating a Blog}\label{creating-a-blog}}
\addcontentsline{toc}{subsection}{Creating a Blog}

Once you have done all the reflections on these 5 steps, you can move forward with creating a blog. Please visit\href{https://create.twu.ca}{create.twu.ca}, click the blue `Log in' link and follow the instructions. Once you are done there, you can return here to continue with the next section to get your site set up.

\hypertarget{wordpress-setup}{%
\section{WordPress Setup}\label{wordpress-setup}}

\hypertarget{wordpress-resources}{%
\subsection*{WordPress Resources}\label{wordpress-resources}}
\addcontentsline{toc}{subsection}{WordPress Resources}

\begin{figure}
\centering
\includegraphics{assets/u1/stephen-phillips-hostreviews-co-uk-sSPzmL7fpWc-unsplash.jpg}
\caption{WordPress Dashboard (\href{https://unsplash.com/photos/sSPzmL7fpWc}{Image Credit})}
\end{figure}

We are here to help you create your site, so do not hesitate to ask for technical support. To get started on creating your site we suggest the following steps:

✔️ Log in if you have not already logged in; get familiar with the \href{http://sites.uci.edu/docs/start/dashboard/}{administration interface} and click \href{https://www.youtube.com/watch?v=-b569fs2t-Y}{here for more information on it}. The administration interface is also where the Dashboard in WordPress is located and \href{https://codex.wordpress.org/Dashboard_Screen}{you can get more info about it here.}\\
✔️ Remember: When you are in the site administration area of your site, you can get tips on what you are doing by clicking the ``Help'' menu on the top-right corner.\\
✔️ Review your settings, start by changing your Site Title under \texttt{Settings\textgreater{}General}. You need to hit the ``Save'' button to save your changes. \href{https://codex.wordpress.org/Settings_General_Screen}{More information about General Settings here.}\\
✔️ \href{http://www.wpbeginner.com/beginners-guide/how-to-add-a-new-post-in-wordpress-and-utilize-all-the-features/}{Add a new post}{]}. You can pick one of the existing categories by checking a box on the sidebar of the authoring interface. \href{http://www.wpbeginner.com/glossary/category/}{You can manage your categories here.} You will need to hit the blue ``Publish'' button on the right hand side before your post appears. \href{https://en.support.wordpress.com/pages/page-visibility/}{Information on managing the privacy settings on individual posts is here.}\\
✔️ Edit your about me page by introducing yourself and sharing a little about yourself.\\
✔️ You are welcome to change the images and upload your own. \href{https://www.youtube.com/watch?v=GtMOAaMFaPs}{Here is information about using images from Google.}

\hypertarget{wordpress-tutorials}{%
\section{WordPress Tutorials}\label{wordpress-tutorials}}

When you're ready to start customizing your blog and putting content in, check out some tutorials available to you:

✔️ \href{http://www.wpbeginner.com/start-here/}{Beginner's guide for WordPress by WPBeginner}\\
✔️ \href{https://learn.wordpress.com/}{Learn WordPress website by WordPress}\\
✔️ \href{http://digitaltattoo.ubc.ca/}{Digital Tattoo Project at UBC} -- learn digital literacy skills. Check out the menu sections and \href{http://digitaltattoo.ubc.ca/publish/}{consider looking at the publish section.}

If you are confused about anything it is always good to do an initial Google or YouTube search, or reach out to your learning pod or your instructor.

\begin{todo}
\textbf{Things To Do This Week}

✔️ Meet in Zoom
\href{https://www.timeanddate.com/worldclock/fixedtime.html?msg=LDRS663+Meeting\&iso=20220325T0730\&p1=1109}{Friday,
March 25 - 7:30 AM PDT (Click to check your local time)} AND
\href{https://www.timeanddate.com/worldclock/fixedtime.html?msg=LDRS663+Meeting\&iso=20220401T0730\&p1=1109\&ah=1\&am=30}{Friday,
April 1 - 7:30 AM PDT (Click to check your local time)}\\
✔️ \textbf{Create} a Visitor/Resident diagram to visualize your online
presence.\\
✔️ \textbf{Register} for a site at
\href{https://create.twu.ca}{create.twu.ca}. I suggest you start with a
`General TWU site'.\\
✔️ \textbf{Post} your site URL in the `WordPress URLs' Glossary in
Moodle.\\
✔️ Please complete these items by \textbf{\emph{Friday April 1}}
\end{todo}

\hypertarget{unit-1-assessment}{%
\subsection{Unit 1 Assessment}\label{unit-1-assessment}}

\href{https://ma-lead.github.io/ldrs663/assessments}{Please see the details for Post 1}

\hypertarget{learning-in-community}{%
\chapter{Learning in Community}\label{learning-in-community}}

\begin{todo}
\hypertarget{things-to-do-this-week}{%
\paragraph{Things To Do This Week}\label{things-to-do-this-week}}

\begin{itemize}
\tightlist
\item
  Meet in Zoom
  \href{https://www.timeanddate.com/worldclock/fixedtime.html?msg=LDRS663+Meeting\&iso=20220408T0730\&p1=1109\&ah=1\&am=30}{Friday,
  April 8 - 7:30 AM PDT (Click to check your local time)}\\
\item
  \textbf{Read}
  \href{https://www-sciencedirect-com.twu.idm.oclc.org/science/article/pii/S1096751600000166}{Critical
  inquiry in a text-based environment: Computer conferencing in higher
  education}\\
\item
  \textbf{Read}
  \href{http://www.irrodl.org/index.php/irrodl/article/view/149/230}{Getting
  the mix right again: An updated and theoretical rationale for
  interaction}\\
\item
  \textbf{Read}
  \href{https://link-springer-com.ezproxy.student.twu.ca/article/10.1007/s12528-011-9049-4}{Interaction
  and the online distance classroom: Do instructional methods effect the
  quality of interaction?}\\
\item
  \textbf{Publish} your arguments for or against the \emph{Interaction
  Equivalency Theorem} in the Moodle \emph{Unit 1 Forum}.\\
\item
  Please complete these tasks by
  \href{https://www.timeanddate.com/worldclock/fixedtime.html?msg=LDRS663+Meeting\&iso=20220408T2300\&p1=1109\&ah=1\&am=30}{\textbf{Friday,
  April 8 @ 11:00 PM PDT}}
\end{itemize}
\end{todo}

\hypertarget{overview-1}{%
\section*{Overview}\label{overview-1}}
\addcontentsline{toc}{section}{Overview}

Welcome to LDRS 663 - \emph{Coaching for Transformational Blended Learning}! In this second unit, we will begin by considering the nature of learning communities through the lens of a model called the \emph{Community of Inquiry (CoI)} (\href{https://www.sciencedirect.com/science/article/pii/S1096751600000166?}{Garrison et al., 2000}; \href{http://www.aupress.ca/index.php/books/120229}{Vaughan et al., 2013}). The CoI model proposes that there are three overlapping components, or presences, to any learning environment; cognitive presence (constructing meaning), social presence (projecting a sense of yourself), and teaching presence (designing and facilitating the learning experience). The CoI model is grounded in a long history of social constructivism which is the idea that learning is fundamentally a social process (\href{https://en.wikisource.org/wiki/My_Pedagogic_Creed}{Dewey, 1897}; \href{https://twu.idm.oclc.org/login?url=http://search.ebscohost.com/login.aspx?direct=true\&db=cat05965a\&AN=alc.191437\&site=eds-live}{Vygotsky, 1978}). We will also consider various modes of interaction in learning environments and how these two models have informed the model of teaching and learning in TWU FAR Centres.

\hypertarget{topics-1}{%
\subsection*{Topics}\label{topics-1}}
\addcontentsline{toc}{subsection}{Topics}

This unit is divided into 3 topics:

\begin{enumerate}
\def\labelenumi{\arabic{enumi}.}
\item
  Introduction to the Community of Inquiry Model
\item
  Modes of Interaction
\item
  Interaction Equivalency Theorem
\end{enumerate}

\hypertarget{learning-outcomes}{%
\subsection*{Learning Outcomes}\label{learning-outcomes}}
\addcontentsline{toc}{subsection}{Learning Outcomes}

When you have completed this unit you should be able to:

\begin{itemize}
\tightlist
\item
  Analyze the characteristics of the Community of Inquiry model.
\item
  Evaluate different modes of interaction.
\item
  Criticize the Interaction Equivalency Theorem.
\end{itemize}

\hypertarget{resources-2}{%
\subsection*{Resources}\label{resources-2}}
\addcontentsline{toc}{subsection}{Resources}

Here are the resources you will need to complete this unit:

\begin{itemize}
\tightlist
\item
  \href{https://www-sciencedirect-com.twu.idm.oclc.org/science/article/pii/S1096751600000166}{Critical inquiry in a text-based environment: Computer conferencing in higher education}\\
\item
  \href{http://www.irrodl.org/index.php/irrodl/article/view/149/230}{Getting the mix right again: An updated and theoretical rationale for interaction}\\
\item
  \href{https://rdcu.be/cKSGf}{Interaction and the online distance classroom: Do instructional methods effect the quality of interaction?}
\end{itemize}

\begin{protip}
\hypertarget{protip}{%
\paragraph*{✨ ProTip}\label{protip}}
\addcontentsline{toc}{paragraph}{✨ ProTip}

Ok\ldots it would seem that the TWU library has lost access to the
database that contains the Kanuka article above (`Interaction and the
online distance classroom\ldots{}'). Here is how I go about tracking
down articles In this case, I've completed all of the checked items and
am waiting to hear back from Dr.~Kanuka):

✅ I see if I have access through another library. As a PhD candidate at
UVic, I have access to their library, and it turns out that this article
is in their collection, but that doesn't help you.\\
✅ I use a Firefox extension called `Unpaywall', which searches for
open-access versions of the same article as researchers are often able
to post pre-print versions of their article on their own site. In this
case, unpaywall just links to the PDF that I have access to through
UVic.\\
✅ I contact the \href{https://libguides.twu.ca/help}{TWU library} for
assistance, as librarians are among the most helpful people on the
planet. In this case, I was provided an open link, but that link looks
fishy as it isn't apparently tied to Kanuka's site, but rather an
aggregation site. Often, I also request an inter-library loan, in which
the TWU library would contact another library (maybe UVic) and request a
copy through them.\\
✅ Search \href{https://scholar.google.com}{Google Scholar} as that will
often pick up on preprints on the author's website.\\
✅ Search \href{https://www.researchgate.net/}{ResearchGate} for
preprints.\\
✅ Contact the author of the paper. They often have permission to share
preprints (see above) and are usually happy to share their work,
especially with grad students.\\
❌ Here is where we start getting into academic grey areas\ldots so
don't do this. It is against publishers policy to use tools like the
hashtag \#ICanHazPDF on Twitter, where you post the title of the article
you are looking for under that hashtag and another researcher might
respond by DM'ing you a copy. You shouldn't do this, because it cuts
into the \textbf{\emph{absurd}} profits of massive multi-national
corporations who benefit from the free work of researchers all over the
world. This is nothing like walking down a hallway and knocking on your
colleague's door to ask if they have a copy of a paper, because that is
super illegal too and this is digital\ldots or something.

And this whole mess of strategies could be avoided if everyone published
under an open license and the research that is created, usually with
government funding, was freely available to those who need it.

\hypertarget{an-update-on-this-link}{%
\subparagraph{An update on this
link\ldots{}}\label{an-update-on-this-link}}

✅ I received a response from the author, who generously provided a PDF
copy of the article. This is very common.\\
✅ Also, I discovered at the bottom of the page when I viewed the
article signed into a different library, that there was an option to get
a sharing link, so I copied that link and updated the URL above. This
seems to provide a read-only link for people who don't have access
through a library or other institution.

\begin{figure}
\centering
\includegraphics{assets/u2/share.png}
\caption{Share this article}
\end{figure}
\end{protip}

\hypertarget{optional-resource}{%
\subsubsection*{Optional Resource}\label{optional-resource}}
\addcontentsline{toc}{subsubsection}{Optional Resource}

\begin{itemize}
\tightlist
\item
  Vaughan, N., Cleveland-Innes, M., \& Garrison, D. (2013). \emph{Teaching in blended learning environments: Creating and sustaining communities of inquiry.} Athabasca: AU Press. - This book is available for free at \href{http://www.aupress.ca/index.php/books/120229}{AUPress}.
\end{itemize}

\hypertarget{introduction-to-the-community-of-inquiry-model}{%
\section{Introduction to the Community of Inquiry Model}\label{introduction-to-the-community-of-inquiry-model}}

Before you dive into the content of this unit in LDRS 663, take a moment to recall some particularly memorable learning experiences that you have had. They don't have to be particularly profound in terms of \emph{what} you learned, but profound because of the fact that you still remember \emph{that} you learned something and \emph{how} you learned it. Pick one or two of those experiences and be prepared share them in our class meeting. Make sure to tell us about the context of your experience. Who was there? What did you do to learn? Why do you remember it?

\hypertarget{social-constructivism}{%
\subsection*{Social Constructivism}\label{social-constructivism}}
\addcontentsline{toc}{subsection}{Social Constructivism}

There is a very good chance that your recollection of a memorable learning experience as part of the previous learning activity included a description of some sort of social interaction. This isn't always the case, but the idea that learning is a social process has a long history in education. Many theorists credit John Dewey for bringing this idea to the forefront of educators' minds. In his 1897 treatise \emph{My Pedagogic Creed} he writes:

\begin{quote}
I believe that the school is primarily a social institution. Education being a social process, the school is simply that form of community life in which all those agencies are concentrated that will be most effective in bringing the child to share in the inherited resources of the race, and to use his own powers for social ends. (p.~7)
\end{quote}

The idea didn't originate with Dewey, though, as we know that in first-century Palestine there was a certain itinerant teacher whose lessons were profoundly impactful on a small group of young men and women who were called to live and learn in a deeply personal and social community.

Following Dewey, many others, such as Jean Piaget, Jerome Bruner, and Lev Vygotsky (see \href{https://twu.idm.oclc.org/login?url=http://search.ebscohost.com/login.aspx?direct=true\&db=cat05965a\&AN=alc.1254633\&site=eds-live}{Driscoll, 2005)} have written about what has now become known as the educational theory of \emph{social constructivism}, or, more concisely, constructivism. Driscoll (2005) describes constructivism as a theory that
\textgreater{} rests on the assumption that knowledge is constructed by learners as they attempt to make sense of their experiences. Learners, therefore are not empty vessels waiting to be filled, but rather active organisms seeking meaning. (p.387)

This process of seeking meaning is an iterative process whereby the learner experiences some sort of cognitive dissonance, or a disconnect between what they previously knew and some new piece of evidence or experience that disconfirms that knowledge. The learner then seeks to resolve that dissonance by either incorporating the new experience into an older schema, or by disregarding one or the other. Most often, the resulting knowledge is constructed from portions of both the new and old idea.

\hypertarget{community-of-inquiry}{%
\subsection*{Community of Inquiry}\label{community-of-inquiry}}
\addcontentsline{toc}{subsection}{Community of Inquiry}

This brings us to the idea of a `Community of Inquiry' (CoI), which was first described by Garrison, Anderson, and Archer in their 2000 article ``Critical Inquiry in a text-based environment''. Garrison, et al.~theorize that there are three critical components, or ``presences'' that compose an interactive, online learning environment: Cognitive presence, social presence, and teaching presence. The intersection of these three presences is the heart of an educational experience.

\begin{figure}
\centering
\includegraphics{assets/u2/CoI-Model.jpg}
\caption{Community of Inquiry Model (Garrison, et al., (2000)}
\end{figure}

\hypertarget{cognitive-presence}{%
\subsubsection*{Cognitive Presence}\label{cognitive-presence}}
\addcontentsline{toc}{subsubsection}{Cognitive Presence}

Cognitive presence, possibly the most foundational element, is the
\textgreater{} extent to which the participants in any particular configuration of a community of inquiry are able to construct meaning through sustained communication (p.~89).

Recall that this cognitive process is at the heart of constructivist learning environments. It seems obvious (be careful when people say that) that this construction of meaning through communication is the entire point of higher education. Your task as a student is to change your own mind, and that is a very tall order as our beliefs about many things are remarkably resilient. The way we engage in this task will have a significant bearing on the outcomes of the task.

If we approach communication with too much confidence in our own views, we can shut out competing ideas to our own detriment, so it is important to bring a cautious intelligence, or, as I once heard a student describe it, epistemic humility. We all know that we are wrong about some things. The trouble is we don't know what we are wrong about and how we have misunderstood.

Cognitive presence in a text-based environment (like an online course) carries with it some affordances, but also some disadvantages. It is a relatively common experience for people to type something in a text message or an email, only to have their intentions grossly misunderstood because there are fewer para-linguistic cues in text compared to verbal face-to-face communication. Recent developments in incorporating emojis have started to change this, but emojis are generally considered to be too informal for `serious scholarly work' or written professional communication. So, the relatively lean environment of text can lead to significant misunderstandings.

On the other hand, for learners like me who tend toward introversion, the asynchronous nature of text-based learning environments is a huge advantage. I couldn't count the number of times that I have wanted to contribute to an in-class discussion, but needed too much time to formulate a coherent response, and before I knew it, the conversation had moved on. My point was no longer relevant, having been resolved by those in the class who were more extroverted and ready with an answer. A text-based environment, however, gives me \emph{time to think, write, revise, and then post} my response.

\hypertarget{social-presence}{%
\subsubsection*{Social Presence}\label{social-presence}}
\addcontentsline{toc}{subsubsection}{Social Presence}

Garrison, et.al. describe social presence as
\textgreater the ability of participants in the Community of Inquiry to project their personal characteristics into the community, thereby presenting themselves to the other participants as ``real people.'' (p.~89)

In any community, and especially the TWU community, the ability to \emph{belong} and to be accepted as a whole and integrated person is critical to people feeling like they \emph{actually do belong}. It is for this reason that many experienced online educators encourage a more colloquial style of writing in online forums or blogs. Strict adherence to APA or other style guides virtually eliminates self-referential language such as personal pronouns. It is hard to project your personal characteristics as a real person when you can only refer to yourself in the third person.

By allowing a more personal style and the projection of self into the community, it is thought that students will build a sense of trust in the community and feel more empowered to participate in the difficult work of changing their minds. Social presence supports cognitive presence by allowing the learning environment to be safe and welcoming.

\hypertarget{teaching-presence}{%
\subsubsection*{Teaching Presence}\label{teaching-presence}}
\addcontentsline{toc}{subsubsection}{Teaching Presence}

This final element of the CoI model is the design and facilitation of the learning experience
\textgreater to support and enhance social and cognitive presence for the purpose of realizing educational outcomes. (p.~90).

Teaching presence can be a shared function between members of the community. Garrison, et al.~point out that the design of the experience is typically performed by the teacher, and the facilitation is more often shared. In a connected course like this one, there is a greater emphasis on shared facilitation in a community of learners compared to what might be experienced in a f2f (face to face) course. It is in this shared discourse in a safe environment that allows learners to engage in the difficult cognitive work of learning.

\hypertarget{learning-activity-1}{%
\subsection*{Learning Activity}\label{learning-activity-1}}
\addcontentsline{toc}{subsection}{Learning Activity}

\begin{reflect}
\hypertarget{read-annotate-and-reflect}{%
\paragraph{Read, Annotate, and
Reflect}\label{read-annotate-and-reflect}}

\textbf{Read}
\href{https://www-sciencedirect-com.twu.idm.oclc.org/science/article/pii/S1096751600000166}{Critical
inquiry in a text-based environment: Computer conferencing in higher
education} (access through the TWU library).

If you haven't done so previously, sign up for and activate hypothes.is
and while you are reading the article, leave some annotations that
connect what you are reading to your own experience.

Use the tag `ldrs663' in any annotations you create so that we can all
find each other.

\href{http://create.twu.ca/help/other-web-tools/hypothesis}{Click here
for assistance getting set up with hypothes.is.}

As you read, consider a time where you experienced a learning
environment where the three presences described in the CoI were
apparent. Consider the following questions and optionally, write your
responses in a new post on your blog. - Were all three presences
demonstrated? - Which of the three were most obvious? Least? - Which
presence is most important for you?

Note that this is an ungraded activity, but is designed to help prepare
you for the assessments in this course. Throughout the course you are
encouraged to take notes in a journal of some sort. Refer to these notes
as you complete your assessments.
\end{reflect}

\hypertarget{modes-of-interaction}{%
\section{Modes of Interaction}\label{modes-of-interaction}}

For this next topic we will look at what we mean by `interaction', a word which is thrown around a lot in educational technology, but\ldots{}

\href{https://youtu.be/G2y8Sx4B2Sk}{Watch}

Before we get into the the topic of interaction, please take a few minutes to answer the following questions about scenarios that may or may not be considered `interaction'.
(Note that you can check your answer right away, and then click the arrow for the next example.)

What do you think? Do you agree with the `correct' and `incorrect' responses on the quiz?

\hypertarget{interaction}{%
\subsection*{Interaction}\label{interaction}}
\addcontentsline{toc}{subsection}{Interaction}

Anderson (2003) argues that, despite the lack of clarity around definitions of interaction, there seems to be a general understanding that interaction of some sort is a requirement for learning. He settles on the definition from Wagner (1994, p.8)

\begin{quote}
reciprocal events that require at least two objects and two actions. Interactions occur when these objects and events mutually influence one another.
\end{quote}

He provides a model of interaction in learning environments that includes three main agents in the process: students, teachers, and content (Figure 1).

\begin{figure}
\centering
\includegraphics{assets/u2/Modes-Interaction-Anderson.png}
\caption{Figure 1. Anderson's Modes of Interaction (2003)}
\end{figure}

At each point of the triangle are the agents in an educative process. The arrows between them indicate the two-way communication described by Wagner, and the recursive arrows above or below the agents are secondary forms of interaction.

\hypertarget{other-models-of-interaction}{%
\subsection*{Other Models of Interaction}\label{other-models-of-interaction}}
\addcontentsline{toc}{subsection}{Other Models of Interaction}

Kanuka (2011) describes a modified model of interaction which presumes that all educational interactions occur in the context of some sort of content (Figure 2.).

\begin{figure}
\centering
\includegraphics{assets/u2/Kanuka-Modes-of-Interaction.png}
\caption{Figure 2. Kanuka's Model of Interaction (2011)}
\end{figure}

In Anderson's model, content is an agent in the process, but in Kanuka's model, content of some sort is assumed to be the foundation of learning environments and that interactions between and among learners and instructors happens in the context of making sense of the content. The content itself does not have agency.

A combination of these two models was described by Madland (2014). Madland's model, shown in figure 3, is a return to Anderson's three-sided model except with the addition of peer interactions, and all interactions between agents occuring in the context of the content that is to be learned. Also added are the three sides of the model representing structured learning activities designed specifically to enhance the educative effects of the interactions.

\begin{figure}
\centering
\includegraphics{assets/u2/Modes-of-Interaction-Madland.png}
\caption{Figure 3. Madland's Model of Interaction (2014)}
\end{figure}

It is not enough to simply expect interactions between learners and instructors to be goal-oriented towards learning outcomes. Educators must design specific activities (teaching presence) to create the conditions (social presence) for learning to occur (cognitive presence). An example of this can be seen in the seemingly ubiquitous `Group Project' assigned in so many undergraduate courses. Sometimes, groups are allowed to form themselves, other times, the instructor assigns groups, or there is some sort of random process to create groups. However they are formed, groups too often fall into a pattern of behaviour where one or two of the students do most of the work and the remaining group members engage in social loafing and benefit from their peers' work.

An alternative practice is to engage groups in a specific structured process of cooperation where everybody must do their work, or else the entire group suffers. An example is to have students work in peer review partners where students submit their assignments to their peer review partner who then provides feedback based on specific questions and categories of comments. When the original students receives their partner's feedback, they may choose how to integrate the suggestions into the final assignment that they submit to their instructor. Each student is then graded on the quality of their finished assignment, the quality of the feedback that they provided, and on the rationale for how they incorporated the peer review into their own work.

There are myriad structures that can be used to ensure that interactions are inclined towards producing student learning, and we will introduce you to some of those in unit 5.

\hypertarget{learning-activity-2}{%
\subsection*{Learning Activity}\label{learning-activity-2}}
\addcontentsline{toc}{subsection}{Learning Activity}

\begin{reflect}
\hypertarget{read-and-reflect}{%
\paragraph{Read and Reflect}\label{read-and-reflect}}

Read the following article by Terry Anderson -
\href{http://www.irrodl.org/index.php/irrodl/article/view/149/230}{Getting
the mix right again: An updated and theoretical rationale for
interaction}

As you read, consider what interactions you have experiences in online
or f2f courses.
\end{reflect}

\hypertarget{interaction-equivalency-theorem}{%
\section{Interaction Equivalency Theorem}\label{interaction-equivalency-theorem}}

For our last topic of this unit, we'll explore Anderson's (2003) \emph{Interaction Equivalency Theorem}, stated as

\begin{quote}
Deep and meaningful formal learning is supported as long as one of the three forms of interaction (student--teacher; student-student; student-content) is at a high level. The other two may be offered at minimal levels, or even eliminated, without degrading the educational experience.
High levels of more than one of these three modes will likely provide a more satisfying educational experience, though these experiences may not be as cost or time effective as less interactive learning sequences. (p.~4)
\end{quote}

At TWU, there has always been a significant emphasis placed on student-teacher interactions. This can be seen in small class sizes and opportunities for students to be involved in faculty-led research projects and travel studies. Distance education, however, has long suffered from a distinct lack of student-teacher and student-student interactions. For many years, distance education was delivered either through the post or through one-way media such as radio or TV, virtually eliminating interactions. This led to a pervasive view that distance education courses and programs were second-rate at best.

However, now that modern communication infrastructure has developed to the point that media-rich, synchronous, two-way communication is almost free, opportunities for distance learning environments to include high levels of student-student and student-teacher interaction are much more feasible.

One problem remains, though, and that is that student-teacher interaction is a scarce commodity. It is costly to hire enough faculty to enable one-on-one or small-group interaction between students and faculty. The distance educator's response to this challenge is to front-load the faculty input (high-level disciplinary expertise or cognitive presence) into the course materials and to de-couple `interaction' from both time and place.

In an asynchronous, text-based environment, students and teachers do not need to be present in the same place at the same time in order to enjoy rich interactions.

\hypertarget{unit-2-assessment}{%
\section*{Unit 2 Assessment}\label{unit-2-assessment}}
\addcontentsline{toc}{section}{Unit 2 Assessment}

\begin{wp}
\hypertarget{wordpress-post}{%
\paragraph{WordPress Post}\label{wordpress-post}}

Please complete the assignment under `Post 2' below. This will be your
first of five draft posts during the course.

\href{https://ma-lead.github.io/ldrs663/assessments.html\#post-2}{Blog
Post 2}
\end{wp}

\hypertarget{references}{%
\section*{References}\label{references}}
\addcontentsline{toc}{section}{References}

Anderson, T. (2003). Getting the mix right again: An updated and theoretical rationale for interaction. \emph{International Review of Research in Open and Distance Learning,} 4(2), 1--14.

Dewey, J. (1897). \emph{My pedagogic creed.} In M. S. Dworkin (Ed.), \emph{Dewey on education.} NewYork, NY: Teachers College Press.

Driscoll, M. P. (2005). \emph{Psychology of learning for instruction} (3rd ed.). Boston: Pearson Education.

Garrison, D. R., Anderson, T., \& Archer, W. (2000). Critical inquiry in a text-based environment: Computer conferencing in higher education. \emph{The Internet and Higher Education, 2}, 87--105. \url{https://doi.org/10.1016/S1096-7516(00)00016-6}

Kanuka, H. (2011). Interaction and the online distance classroom: Do instructional methods effect the quality of interaction? \emph{Journal of Computing in Higher Education,} 23(2), 143--156. \url{https://doi.org/10.1007/s12528-011-9049-4}

Madland, C. (2014). \emph{Structured student interactions in online distance learning: Exploring the study buddy activity} (Master's thesis). Athabasca University. Retrieved from \url{http://hdl.handle.net/10791/47}

\hypertarget{learning-facilitation}{%
\chapter{Learning Facilitation}\label{learning-facilitation}}

\begin{todo}
\hypertarget{things-to-do-this-week}{%
\paragraph{Things To Do This Week}\label{things-to-do-this-week}}

\begin{itemize}
\tightlist
\item
  Meet in Zoom
  \href{https://www.timeanddate.com/worldclock/fixedtime.html?msg=LDRS+663+Meeting\&iso=20220422T0730\&p1=1109\&ah=1\&am=30}{Friday,
  April 22 - 7:30 AM PDT}\\
\item
  \textbf{Read}
  \href{https://infed.org/mobi/facilitating-learning-and-change-in-groups-and-group-sessions/}{\textbf{Facilitating
  Learning and Change in Groups}}\\
\item
  \textbf{Read}
  \href{https://infed.org/mobi/what-is-a-group/}{\textbf{What is a
  Group?}}\\
\item
  \textbf{Read}
  \href{https://www.researchgate.net/publication/228957278_From_Comfort_Zone_to_Performance_Management}{\textbf{Comfort
  Zone to Performance Management}}\\
\item
  \textbf{Read}
  \href{https://www.iaf-world.org/site/professional/core-competencies}{\textbf{Core
  Competencies}}\\
\item
  \textbf{Watch}
  \href{https://player.vimeo.com/video/364868276}{Liberating
  Structures}\\
\item
  \textbf{Visit}
  \href{http://www.liberatingstructures.com/9-what-so-what-now-what-w/}{\textbf{What,
  So What, Now What? W³}}\\
  ✅ Please complete these items by
  \href{https://www.timeanddate.com/worldclock/fixedtime.html?msg=LDRS+663+Meeting\&iso=20220422T0730\&p1=1109\&ah=1\&am=30}{Friday,
  April 22 - 7:30 AM PDT}
\end{itemize}
\end{todo}

\begin{protip}
\hypertarget{note}{%
\paragraph*{✨ NOTE}\label{note}}
\addcontentsline{toc}{paragraph}{✨ NOTE}

Your \href{https://far.twu.ca/ldrs/663-202109/assignments}{\textbf{Small
Group Facilitation Session}} is due \textbf{\emph{Friday, May 6}}.
Please begin scheduling your facilitation sessions with your groups this
week so you have enough time to complete the sessions and your
reflections.

Also, you should be planning your
\href{https://ma-lead.github.io/ldrs663/assessments.html\#facilitated-curriculum-analysis-10}{\emph{Curriculum
analysis}} assignment with your learning pod.
\end{protip}

\hypertarget{overview-2}{%
\subsection*{Overview}\label{overview-2}}
\addcontentsline{toc}{subsection}{Overview}

Facilitation in education refers to the process of helping learners to explore, learn and change. A facilitator is expert on process and group interactions. In education, facilitation is rooted in understanding the nature of the social learning process and how to guide its direction and quality. As a social species, we learn a great deal from each other in both formal and informal contexts. Our earliest learning experiences are profoundly social and intimate interactions between mother and child, and the social aspect of learning never ceases to be important. During this unit, we will examine a short history of social theories of learning from John Dewey and Lev Vygotsky, then, we will experiment with the theory and practices of facilitating learning in group settings.

\hypertarget{topics-2}{%
\subsection*{Topics}\label{topics-2}}
\addcontentsline{toc}{subsection}{Topics}

This unit is divided into the following topics:

\begin{enumerate}
\def\labelenumi{\arabic{enumi}.}
\tightlist
\item
  Social Theories of Learning
\item
  Cooperative Learning
\item
  Facilitating Transformational Learning in Group Settings
\item
  Navigating Group Dynamics
\item
  Core Facilitation Competencies
\item
  Strategies for Learning Facilitation
\end{enumerate}

\hypertarget{learning-outcomes-1}{%
\subsection*{Learning Outcomes}\label{learning-outcomes-1}}
\addcontentsline{toc}{subsection}{Learning Outcomes}

When you have completed this unit, you should be able to:

\begin{itemize}
\tightlist
\item
  Explain how to design learning environments to maximize learning
\item
  Plan appropriate group learning processes to support transformative learning.
\item
  Demonstrate how to facilitate a course of study.
\item
  Design cooperative activities to maximize student-student and student-content interactions
\item
  Apply knowledge of the Community of Inquiry model and liberating structures to the facilitation of cooperative learning activities
\item
  Identify and explain core competencies for facilitating learning.
\end{itemize}

\hypertarget{resources-3}{%
\subsection*{Resources}\label{resources-3}}
\addcontentsline{toc}{subsection}{Resources}

Online resources will be provided in the unit.

\hypertarget{social-theories-of-learning}{%
\section{Social Theories of Learning}\label{social-theories-of-learning}}

The idea that learning is a social process can be traced way back in time, but formal descriptions of social constructivism, as it has been called, are often traced to John Dewey, Jean Piaget, and Lev Vygotsky. Albert Bandura also contributed via social learning theory. \emph{What is social constructivism?}

Vygotsky (1978) argues that ``every function in the child's cultural development appears twice: first, on the social level, and later on the individual level; first \emph{between} people (\emph{interpsychologically}), and then inside the child (\emph{intrapsychologically}) (p.~57). That is, humans learn first through social observations and interactions, which they later internalize as their own thinking. The significance of Vygotsky's insight is that, ``instead of focusing on the study of psychological entities such as skills, concepts, information-processing units, reflexes, or mental functions, it assumes that we must begin with a unit of \emph{activity''} (Wertsch, 1985). This idea of \emph{activity,} or to be more precise the active participation of the learner, is central to our emerging understanding of learning as a process of socially constructing knowledge.

Central to this progress is language, which we first learn in the context of social interactions, then, we adopt as self-talk for self-direction and self-regulation, and ultimately internalized as our own inner speech (Vygotsky \& Kozulin, 1986, p.~228). Similarly, Bandura (1977) argues social role modeling is central to how most behaviors are learned, ``from observing others one forms an idea of how new behaviors are performed, and on later occasions this coded information serves as a guide for action''.

In sum, we may conclude that knowledge is \emph{constructed} first socially, then, personally, by learners as they encounter new information, compare it to old models that they may have, and develop new understandings of how the world works. It is not the case that new knowledge is simply copied intact from one mind to another, rather new information is integrated into old understandings, bringing about a hybrid of the two. This is essentially what we understand as a \emph{constructivist} model of learning.

\hypertarget{zone-of-proximal-development}{%
\subsection*{Zone of Proximal Development}\label{zone-of-proximal-development}}
\addcontentsline{toc}{subsection}{Zone of Proximal Development}

Social constructivism builds on the constructivist model, adding the idea that this process of integrating new understandings with old understandings is best understood as a \emph{social} process. Vygotsky (1978) introduced the idea that people with a greater capacity to understand the world and cope with its challenges act as supportive structures, which enable others to \emph{construct} and \emph{internalize} the knowledge these people have. This new construction occurs within what Vygotsky refers to as an individual's \textbf{\emph{zone of proximal development,}} which he differentiated from their \textbf{\emph{zone of actual development.}} The ZPD is the `sweet spot' in education where a student is optimally challenged to learn. If the task is too easy for the student, and they have already mastered it, then learning activities will not result in learning. Conversely, if a task is so difficult that the student cannot complete it, even with assistance, learning activities will not result in learning. In the middle are tasks that a student is able to complete, but only with the assistance of a more capable peer or expert. This is the Zone of Proximal Development:

\begin{figure}
\centering
\includegraphics{assets/u3/ZPD_Image.png}
\caption{Zone of Proximal Development}
\end{figure}

\hypertarget{scaffolding}{%
\subsection*{Scaffolding}\label{scaffolding}}
\addcontentsline{toc}{subsection}{Scaffolding}

A metaphor that has been used to describe one such supportive mechanism is \emph{scaffolding.} A scaffold is a way for educators to support the construction of new knowledge, beginning from a person's existing repertoire of knowledge and then preceding into new heights of understanding. The scaffold is the environment an educator creates, the support of learning facilitation, and the processes and language that are lent to the learner in the context of approaching an adaptive task and developing new abilities to meet it (Wilhelm, Baker, \& Dube, 2001). Furthermore, scaffolding implies not only a person's specific relation to the modeled behavior of others; it implies a person's relation to social communities, and ``it implies becoming a full participant, a member, a kind of person'' (Lave \& Wenger, 1996).

The task of facilitating learning with the ZPD in mind assumes that you, as the facilitator, know what knowledge and skills your students are starting with. It is likely that the competencies your students display fall along a bell curve. While most students will be learning within a similar ZPD, there will be outliers at both ends of the curve. One strategy can be to pair students whose skills and knowledge are below the curve with those above the curve. In doing this, students who are above the curve, who may not be challenged with the content or skill, have to engage in greater levels of cognitive complexity in order to concisely explain to their peer and help them to meet the objective. For example, Madland and Richards (2016) describes a cooperative peer review activity, where the authors asked a group of learners how they thought cooperative learning activities supported learning. In this study, the learners' responses indicated that the two most important factors were \emph{social cohesion} and \emph{developmentally appropriate challenges,} indicating that learners recognized the importance of the ZPD in learning.

\begin{reflect}
\hypertarget{learning-activity}{%
\paragraph{Learning Activity}\label{learning-activity}}

\textbf{\emph{Questions to Consider}}\\
After reading the topic above, consider the following question:\\
- How does the idea of the zone of proximal development help you support
learning?
\end{reflect}

\hypertarget{cooperative-learning}{%
\section{Cooperative Learning}\label{cooperative-learning}}

Cooperative learning is a set of learning facilitation strategies that are focused on encouraging educative social interactions between learners. It is important to not conflate \emph{cooperative learning} with \emph{group projects} as you might remember them from your previous experiences as a university student. Group projects are often assigned because faculty seem to have a sense that \emph{working together} is a good thing for students, along with a vague sense that modern jobs all require teamwork. Too often, they amount to repurposing an individual assignment (such as, a research paper) into the same task, but with multiple people handing in one item instead of three to four. When these tasks are not well structured, the process becomes problematic.

\begin{figure}
\centering
\includegraphics{assets/u3/U5_T2_Image.jpg}
\caption{Group Projects}
\end{figure}

We have all likely experienced less-than-ideal group projects where one or two people do most of the work, one member is seemingly absent altogether, and another's work is of poor quality. This is not the kind of learning activity that inspires highly engaged learners.

Contrary to this dysfunctional group learning model, cooperative learning is structured in a way that maximizes effort from all students and, ideally, leads to all group members attaining high-level learning outcomes. In order to ensure this, there are five characteristics of learning groups that must be present for cooperative learning to occur: \emph{``positive interdependence, individual accountability, promotive interactions, appropriate use of social skills, and group processing''} (Johnson \& Johnson, 2009, p.~366).

\hypertarget{more-about-cooperative-learning}{%
\subsubsection*{More about cooperative learning\ldots{}}\label{more-about-cooperative-learning}}
\addcontentsline{toc}{subsubsection}{More about cooperative learning\ldots{}}

\hypertarget{positive-interdependence}{%
\paragraph*{Positive Interdependence}\label{positive-interdependence}}
\addcontentsline{toc}{paragraph}{Positive Interdependence}

Positive interdependence, according to Johnson and Johnson is the idea that individuals in a learning environment are dependent upon each other for success. In other words, I cannot succeed unless you succeed and you cannot succeed unless I succeed. So, collectively, we are interdependent. Positive interdependence is the key that distinguishes cooperative learning from competitive learning, where students are graded on a curve and only the top 2-3\% of students can earn `A' grades.

\hypertarget{individual-and-group-accountability}{%
\paragraph*{Individual and Group Accountability}\label{individual-and-group-accountability}}
\addcontentsline{toc}{paragraph}{Individual and Group Accountability}

In cooperative learning environments, each individual in the group is held accountable for their contributions to the final product, and feedback is provided to both the individual and the group. This helps to ensure that students who need more assistance are identified and can be supported as needed, and it also prevents the `social loafing' that is common in typical `group projects.'

\hypertarget{promotive-interaction}{%
\paragraph*{Promotive Interaction}\label{promotive-interaction}}
\addcontentsline{toc}{paragraph}{Promotive Interaction}

Promotive interaction is the logistics of working and learning together as a cooperative group. The essence is that group members each need to work to promote the learning of each other member of the group. Since each person will be held accountable for their work and the entire group will only succeed if each member succeeds, there is a natural social pressure on more experienced members of the group to assist those with less experience or knowledge.

\hypertarget{interpersonal-skills}{%
\paragraph*{Interpersonal Skills}\label{interpersonal-skills}}
\addcontentsline{toc}{paragraph}{Interpersonal Skills}

Not only do members of the group need to learn the content of the lesson or project, but they must also learn the process of working well as a cooperative group. Sometimes, these processes need to be taught directly, other times (like in graduate studies) it is reasonable to presume that group members will already possess and be willing to utilize effective social skills.

\hypertarget{group-processing}{%
\paragraph*{Group Processing}\label{group-processing}}
\addcontentsline{toc}{paragraph}{Group Processing}

Finally, the group must be able to monitor their process with the goal of improving their work process and product. This metacognitive task is crucial to the long-term improvement and progress towards learning goals.

\begin{figure}
\centering
\includegraphics{assets/u3/Image3.png}
\caption{Cooperative Learning (Slavin, 2011)}
\end{figure}

\begin{reflect}
\hypertarget{learning-activity}{%
\paragraph*{Learning Activity}\label{learning-activity}}
\addcontentsline{toc}{paragraph}{Learning Activity}

\textbf{\emph{Questions to Consider}}\\
After reading through the content in Topic 2, please consider the
following questions:\\
- How can coaching concepts be applied to helping learners learn?\\
- What key characteristics define effective coaching for learning?\\
- How can educators coach learners through the transition of making a
change?
\end{reflect}

\hypertarget{facilitating-transformational-learning-in-group-environments}{%
\section{Facilitating Transformational Learning in Group Environments}\label{facilitating-transformational-learning-in-group-environments}}

In Unit One, we examined the CoI model and identified how teaching presence helps to support both the cognitive and social presences within the educational experience of a course of study. An important idea we emphasized was that teaching presence can be a shared function between members of the learning community and that facilitation of the learning process is often shared (Garrison, et al., 2010). Now, we are interested in examining what a division of the teaching presence might look like if we professionalize the function of learning facilitation within a distributed model of teaching presence.

Our prototype for exploring this model is TWU's own Facilitated Academic Resource (FAR) centre. The facilitation of courses in the TWU FAR Centre model is unique, as you know. From the perspective of a traditional, campus-based faculty member in Langley, a FAR Centre course is an \emph{online} course. The faculty member has worked in the role subject matter expert with an instructional designer to \emph{structure} a course which integrates everything required to create an online community of inquiry with allowances for all three presences: social, cognitive, and teaching. The courses are deployed through online technology and materials are accessed digitally in remote locations. Furthermore, students submit their work to the faculty member who then assesses their work and provides both formative and summative feedback as appropriate.

From the perspective of the remote student, however, the course is much more like a typical F2F course where they are meeting with a group of their fellow students in regularly scheduled learning labs in a central location and are guided through the learning materials by an experienced facilitator.

The rationale for this model is that international students often experience difficulties completing online courses from Western universities, so TWU is providing a F2F Academic Facilitator to support remote students in their individual and group studies through the courses. You, as the Academic Facilitation Specialist, are a critical component of this model. Your skills in coaching facilitating student learning through courses where you may not be a subject matter expert are going to be extremely important.

As such, you will need to start thinking about \emph{how} to facilitate your students' experience of a course of study's learning activities without the immediate F2F presence of a faculty member. In the activity below, you will read about the concept and practices of facilitating transformation learning.

\hypertarget{facilitating-group-learning-sessions}{%
\subsection*{Facilitating Group Learning Sessions}\label{facilitating-group-learning-sessions}}
\addcontentsline{toc}{subsection}{Facilitating Group Learning Sessions}

The work of facilitating the learning process within a group setting begins with an effective plan. Smith (2009) proposes a simple model: EFFECT. This model reminds the facilitator to think about the learning \textbf{environment,} the \textbf{focus} (or purpose) of the session, \textbf{feelings} the session is likely to evoke, \textbf{experiences} learners will explore, \textbf{changes} learners will make as a result of the session, and the \textbf{timings} allocated for all the learning experiences and activities. Next, it's important for the facilitator to plan out the structure of each learning session, which like a story, should have beginnings, middles, and endings. Each stage has a particular task. The beginning encourages learners to explore, the middle engages learners with the subject, and the ending enables learners to move on in their personal learning journey. Drawing upon Evans' (2007) guidelines for helping conversations, Smith (2009) advises facilitators to think about ``the exploration as the first quarter of the session; engaging with the subject and developing understanding as the middle half; and enabling action and development as the final quarter.''

\begin{reflect}
\hypertarget{learning-activity}{%
\paragraph*{Learning Activity}\label{learning-activity}}
\addcontentsline{toc}{paragraph}{Learning Activity}

Read the following article:

\href{https://infed.org/mobi/facilitating-learning-and-change-in-groups-and-group-sessions/}{\textbf{Facilitating
Learning and Change in Groups}}

\emph{Questions to Consider}\\
After completing the reading above, consider the following questions:\\
- According to Roger Schwarz what is a facilitator's main task?\\
- According to Carl Rogers what are the core conditions for facilitating
learning?\\
- What are the three foci of the facilitator role?\\
- What are the core values informing facilitation?\\
- How can the EFFECT model help you to plan a facilitated learning
session?\\
- How does a facilitator effectively structure a facilitated learning
session?
\end{reflect}

\hypertarget{navigating-group-dynamics}{%
\section{Navigating Group Dynamics}\label{navigating-group-dynamics}}

In this Unit we have examined how individual learning may be understood as a social process, however, it is also important for facilitators to understand that the group as a whole also learns and develops.

Forsyth (2017) defines a social group as \emph{``two or more individuals who are connected to one another by and within social relationship''} (p.~3). The number of group members, the presence of links between members, and the nature of those links all shape the group and the quality of learning it supports.

An essential requirement for learning in a group is a sufficient level of cohesion and trust between members. Some of the group characteristics that tend to cultivate trust within a group include:

\textbf{Similarity} - The more similar members are in terms of age, sex, education, skills, attitudes, values, and beliefs, the more likely the group will bond.

\textbf{Stability} - The longer a group stays together, the more cohesive it becomes.

\textbf{Size} - Smaller groups tend to have higher cohesion.

\textbf{Support} - Coaching and encouragement to support other members strengthens the group's identity.

\textbf{Satisfaction} - How pleased group members are with each other's performance, behaviour, and conformity to group norms increases cohesion.

\emph{One factor that tends to erode trust within a group is:}

\textbf{Social Loafing} - There is a tendency of individuals to put in less effort when working in a group context. As group size grows, this effect becomes larger.

\hypertarget{group-development}{%
\subsection*{Group Development}\label{group-development}}
\addcontentsline{toc}{subsection}{Group Development}

It is important for facilitators to understand that groups change over time. One of the most influential and helpful models of group development was articulated by Bruce W. Tuckman (1965). His research finding was that groups typically move through five critical stages of development:

\textbf{Forming:} Members get to know one another, exchange personal information, and establish new relationships.

\textbf{Storming:} Members open up to each other and confront each other's ideas and perspectives.

\textbf{Norming:} Members achieve a consensus about goals, definition of roles, and clear coordination of effort.

\textbf{Performing:} The group is able to function as a unit as they find ways to get the job done smoothly and effectively without inappropriate conflict or the need for external facilitation.

\textbf{Adjourning:} At some point the group ends.

\begin{protip}
\hypertarget{protip}{%
\paragraph*{✨ ProTip}\label{protip}}
\addcontentsline{toc}{paragraph}{✨ ProTip}

\textbf{\emph{Each stage of group development requires the facilitator
to employ variations in their approach.}}
\end{protip}

\hypertarget{transforming-me-to-we}{%
\subsubsection*{Transforming (Me to We)}\label{transforming-me-to-we}}
\addcontentsline{toc}{subsubsection}{Transforming (Me to We)}

\begin{longtable}[]{@{}
  >{\centering\arraybackslash}p{(\columnwidth - 2\tabcolsep) * \real{0.6250}}
  >{\raggedright\arraybackslash}p{(\columnwidth - 2\tabcolsep) * \real{0.3750}}@{}}
\toprule
\begin{minipage}[b]{\linewidth}\centering
Development Phase
\end{minipage} & \begin{minipage}[b]{\linewidth}\raggedright
Facilitation Approach
\end{minipage} \\
\midrule
\endhead
Forming (Unwilling and unable) & Clear goals, \textbf{directions}, fairness, firmness. \emph{(Why? Overcoming denial of the new reality.)} \\
Storming (Willing and unable) & As above, plus \textbf{encouraging participation}, calmness, recognition of concerns. \emph{(Why? Overcoming defence of the old reality.)} \\
Norming (Unwilling and able) & Encouraging, \textbf{confidence building}, clear goals, holding accountable) \emph{(Why? Helping discard the old reality.)} \\
Performing (Willing and able) & Clear goal setting, monitoring, sptrategic preparation, seeking innovative approaches, \textbf{empowering team members}. \emph{(Why? Helping make new adaptations.)} \\
Reforming (Disengaging) & \textbf{Establishing new goals}, solving confusion, managing risks. \emph{(Why? Challenging the new comfort zone.)} \\
\bottomrule
\end{longtable}

\begin{reflect}
\hypertarget{learning-activity}{%
\paragraph*{Learning Activity}\label{learning-activity}}
\addcontentsline{toc}{paragraph}{Learning Activity}

Take a moment to read the following article:

\href{https://infed.org/mobi/what-is-a-group/}{\textbf{What is a
Group?}}

\emph{Questions to Consider}\\
After reading the article above, consider the following questions:\\
- What are some of the key benefits and dangers of learning in group
settings?\\
- What are some key dimensions of groups?\\
- What are the stages of group development?
\end{reflect}

\begin{reflect}
\hypertarget{learning-activity}{%
\paragraph*{Learning Activity}\label{learning-activity}}
\addcontentsline{toc}{paragraph}{Learning Activity}

Take a moment to read the following article:

\href{https://www.researchgate.net/publication/228957278_From_Comfort_Zone_to_Performance_Management}{\textbf{Comfort
Zone to Performance Management}}

\emph{Questions to Consider}\\
After completing the reading above, consider the following questions:\\
- How is White's Optimal Performance Zone similar Vygotsky's ZPD?\\
- How does White's model help a facilitator determine how to adjust
their facilitation strategies?\\
- What insight does White's model provide about how to sustain learning
performance?
\end{reflect}

\hypertarget{core-facilitation-competencies}{%
\section{Core Facilitation Competencies}\label{core-facilitation-competencies}}

The professionalization of learning facilitation within educational settings is a promising, but new development. The core competencies are still emerging, as institutions begin to prototype this model. Below is a tentative list of competencies we have identified through in our preliminary experiments.

\begin{itemize}
\tightlist
\item
  Develop multi-session study plans for completing courses
\item
  Select clear study methods and learning activities
\item
  Prepare time and space to support group learning
\item
  Create and sustain a participatory transformative learning environment
\item
  Guide Group to meet each course learning outcome
\item
  Directing processes for sharing peer feedback (in self-directed learning)
\item
  Providing learners with formative feedback
\item
  Mediating exchange of coursework and feedback between students \& instructor
\end{itemize}

\begin{center}\rule{0.5\linewidth}{0.5pt}\end{center}

\begin{reflect}
\hypertarget{learning-activity}{%
\paragraph{Learning Activity}\label{learning-activity}}

Take some time to read the following article:

\href{https://www.iaf-world.org/site/professional/core-competencies}{\textbf{Core
Competencies}}

\textbf{\emph{Questions to Consider}}\\
After completing the reading above, consider the following questions:\\
- What general facilitation competencies apply to facilitating
learning?\\
- What competencies do you feel are strength areas? What areas do you
need to develop?\\
- How can facilitation skills help you support learner success in an
educational setting?
\end{reflect}

\hypertarget{facilitation-strategies}{%
\section{Facilitation Strategies}\label{facilitation-strategies}}

Strategies for facilitating learning are as numerous and varied as the educators who create them. In the FAR model of professionally facilitated learning we are proposing in this course, each FAR course you may help facilitate in the future has a facilitator's guide that provides designs for each learning activity. While these designs provide you with the majority of the learning facilitation strategies required in a given course, the needs of learners are not always predictable and emergent strategies may be needed.

\hypertarget{liberating-structures}{%
\subsection*{Liberating Structures}\label{liberating-structures}}
\addcontentsline{toc}{subsection}{Liberating Structures}

It can often be challenging to devise new ways of interacting in F2F learning environments, but there are many resources available to facilitators both online and in print. One of those resources is a book and website called \emph{Liberating Structures} which describes a set of 33 structured activities that you can use in your learning labs to generate conversation without resorting to the same old tired `brainstorm.'

\begin{reflect}
\hypertarget{learning-activity}{%
\paragraph{Learning Activity}\label{learning-activity}}

Watch the video below for a quick introduction to \emph{Liberating
Structures:}

Next, visit the \emph{Liberating Structures} website and take a look at
the following activity:

\href{http://www.liberatingstructures.com/9-what-so-what-now-what-w/}{\textbf{What,
So What, Now What? W³}}

Now, consider how could you use this Liberating Structure to guide a
group discussion that would help learners learn a Unit?
\end{reflect}

\hypertarget{unit-3-assessment}{%
\section*{Unit 3 Assessment}\label{unit-3-assessment}}
\addcontentsline{toc}{section}{Unit 3 Assessment}

\begin{wp}
\hypertarget{wordpress-post}{%
\paragraph{WordPress Post}\label{wordpress-post}}

Please complete the assignment under `Post 3' below. This will be your
third of five draft posts during the course.

\href{https://ma-lead.github.io/ldrs663/assessments.html\#post-3}{Blog
Post 3}
\end{wp}

\hypertarget{checking-your-learning}{%
\subsection*{Checking Your Learning}\label{checking-your-learning}}
\addcontentsline{toc}{subsection}{Checking Your Learning}

Before you move on to the next unit, you may want to check to make sure that you are able to:

✅ Explain how to design learning environments to maximize learning.\\
✅ Plan appropriate group learning processes to support transformative learning.\\
✅ Demonstrate how to facilitate a course of study.\\
✅ Design cooperative activities to maximize student-student and student-content interactions.\\
✅ Apply knowledge of the Community of Inquiry model and liberating structures to the facilitation of cooperative learning activities.\\
✅ Identify and explain core competencies for facilitating learning.

\hypertarget{coaching-for-learning}{%
\chapter{Coaching for Learning}\label{coaching-for-learning}}

\hypertarget{overview-3}{%
\section{Overview}\label{overview-3}}

An enduring principle of educational practice is learners learn in different ways and at different rates. Effective educators recognize the uniqueness of each individual learner. They understand each learner arrives at a new learning situation with different lived experiences, existing knowledge, cognitive processing strengths and weaknesses, learning preferences, values, and goals. At the same time, effective educators also recognize some aspects of the learning process are common to most, if not all, learners. For instance, most educators believe learning is an ongoing, reflective process. We tend to broadly believe human are constantly taking in new information, integrating it into what we already know, and developing ideas based on this merging of past and present. In the midst of educator's diverse and sometimes contradictory ideas about learning, there seems to be a great deal of mystery, and countless myths about how the learning process actually works. Unit 4 will introduce some important questions about learning, such as:

❓ How can we as educators ensure our students are truly achieving the outcomes we intend?\\
❓ What practises should educators use in order to create an effective learning environment?\\
❓ What role do learners play in generating and sustaining this environment?\\
❓ How can we maximize the likelihood of transformational learning?

\hypertarget{topics-3}{%
\subsection{Topics}\label{topics-3}}

This unit is divided into the following topics:

\begin{enumerate}
\def\labelenumi{\arabic{enumi}.}
\tightlist
\item
  Theories of Learning (or \emph{What do we Really Know About How People Learn})
\item
  The Practice of Coaching
\item
  Core Coaching Competencies
\end{enumerate}

\hypertarget{learning-outcomes-2}{%
\subsection{Learning Outcomes}\label{learning-outcomes-2}}

When you have completed this unit, you should be able to:

\begin{itemize}
\tightlist
\item
  Describe theories about how people learn.
\item
  Explain how to design learning environments to maximize learning.
\item
  Explain the coaching for learning model.
\item
  Identify essential coaching for learning competencies.
\end{itemize}

\hypertarget{resources-4}{%
\subsection{Resources}\label{resources-4}}

Online resources will be provided in the unit.

\url{https://far.twu.ca/ldrs/663-202103/presentations/coaching\#/coaching-0}

Coaching

Coaching is a way of being, listening, asking, and speaking that draws out and augments characteristics and potential that are already present in a person. \textasciitilde Gallwey, 1997 Every block of stone has a statue inside it and it is the task of the sculptor to discover it.''

{[}ui-accordion independent=true open=none{]}
{[}ui-accordion-item title=``Topic 1''{]}
\includegraphics{https://upload.wikimedia.org/wikipedia/commons/0/05/France_in_XXI_Century._School.jpg}
\emph{This image is in the Public Domain.}

The image you see here was created in the early 20th century in France as a part of a series of images predicting what life would be like in 2000. You can see the full collection \href{https://publicdomainreview.org/collections/france-in-the-year-2000-1899-1910/}{\textbf{here}}.

As you can see, the artist envisioned a process whereby information was fed into a machine of some sort and just automagically downloaded into student minds. Kind of like this scene from the movie the \emph{Matrix}:

\href{https://youtu.be/6vMO3XmNXe4}{plugin:youtube}

These are both examples of how the modern condition of thought (Unit 2), has shaped our thinking about learning. Many modern era educators viewed people as learning ``machines'' who could be ``programmed'' with instructions, just like we program computers. The legacy of this thinking still lingers and the educational model of instructors instructing students through instructions is still prevalent.

In these modern visions, learning is something done \emph{to} students, not something done \emph{by} students. The learner is completely passive in the process. The dreams of grade-schoolers to be able to rest a book on their head and learn by osmosis are just as far-fetched as the prognostications from the early 20th century, from the \emph{Matrix,} and from the claims of many edtech companies today. However, we need to recognize that learning is hard work. Consider Erasmus's (1965) comment in his dialogue \emph{The Art of Learning}, ``For my part, I know no other art of learning than hard work, devotion, and perseverance'' (p.~461). This statement remains as true today, as it was in 1529.

Take a few minutes to watch this video from Destin Sandlin, creator of the YouTube channel \emph{Smarter Every Day}. The video has a long introduction, but it is entertaining and sets up the problem of learning very well:

\href{https://youtu.be/MFzDaBzBlL0}{plugin:youtube}

Destin talks about some important ideas related to pedagogy (the study of how children learn) that are also applicable to andragogy (the study of how adults learn). One of those ideas is that \emph{knowledge != understanding} (and now you know that the `!' negates the `=' in that statement). A little later in this unit, when we talk about deep and surface approaches to learning, we will come back to this idea.

!!! As a way to consolidate your own learning around this concept, take a few minutes to think about a time when you have had knowledge, but not understanding, and how that affected you in some way. You don't have to write anything down, just connect the idea to an experience that you have had.

In the video, Destin also talks about cognitive bias and the tremendously powerful effect that our own biases have on what and how well we learn. The challenge that he faced in learning to ride the backwards brain bike wasn't so much the physical skills involved, but \emph{unlearning} his previous `ride a bike' algorithm. When he encountered new information that didn't match his previous experience, it was extremely difficult to overcome his previous experience and bias. It took him 8 months of dedicated effort to do it. At the end, he discovered that he hadn't \emph{overcome} his bias, he had only transferred it to a new bias, albeit one that was easier to overcome.

This process is also true for people learning new ideas. None of us come to a learning context without preconceived ideas regarding the nature of the topic to be studied, and often, those ideas are incorrect. They are misconceptions. Biases. Then, when new information comes along, our brain actively sabotages us by telling us that we already know this and we don't need to listen, while simultaneously convincing us even more of the truth of our misconceptions.

Below is another video, this time from \emph{Veritasium,} in which Derek Muller shows why misconceptions (biases) can be so difficult to overcome:

\href{https://youtu.be/eVtCO84MDj8}{plugin:youtube}

Despite the difficulty, it is important to note that we \emph{can} overcome these biases and misconceptions. In fact, we can overcome some bizarre inputs, as \href{https://www.theguardian.com/education/2012/nov/12/improbable-research-seeing-upside-down}{researchers have discovered and was briefly reported in The Guardian}.

Here is a slightly less scientifically rigorous portrayal of what happens when our vision is artificially flipped: (The video is 12 minutes, but you don't need to watch it all).

\href{https://www.youtube.com/watch?v=OJTC_E2Nlgg}{plugin:youtube}

The point of these experiments, at least for our purposes, is that human brains (and likely other animals as well) demonstrate the characteristic of \emph{plasticity}. The connections in our brain that allow ideas to form and be adjusted can be broken and then reconnected in different configurations.

\textbf{\emph{This is learning. Learning is hard work.}}

One implication of this is that it is naïve for educators to assume that a single exposure to an idea will somehow cause learning. In fact, the opposite is true. It takes repeated exposure, practice, error-correction, and adjustment for learning to become consolidated.

{[}/ui-accordion-item{]}

{[}ui-accordion-item title=``Learning Activity''{]}
\#\#\#\# :fa-book: Read and Reflect

Read Chapters 1-3 of How people learn II: Learners, Context and Cultures, \href{https://www.nap.edu/catalog/24783}{available for purchase or free download here}.

Note: We will not be using this entire book, so don't feel obliged to purchase a copy for yourself.
Click `Read Online' As you read, please use hypothes.is to both record your thoughts and connections in the article, but also interact with past annotations. These chapters are already heavily annotated, presumably by other students.

\textbf{\emph{Questions to Consider\ldots{}}}

After completing the reading above, consider the following questions:

\begin{itemize}
\tightlist
\item
  \emph{What are the basic types of learning?}
\item
  \emph{How does the human brain respond to learning?}
\end{itemize}

{[}/ui-accordion-item{]}

{[}ui-accordion-item title=``Learning Activity''{]}
\#\#\#\# :fa-book: Read and Reflect

Below is a resource that summarizes existing research from cognitive science related to how students learn. This research has practical implications for teaching and learning that will be of benefit as we move forward with the content of this unit. Follow the link below:

\begin{itemize}
\tightlist
\item
  \href{https://deansforimpact.org/wp-content/uploads/2016/12/The_Science_of_Learning.pdf}{\textbf{The Science of Learning}}
\end{itemize}

\textbf{\emph{Questions to Consider\ldots{}}}

After completing the reading above, consider the following questions:

\begin{itemize}
\tightlist
\item
  \emph{How do students understand new ideas?}
\item
  \emph{How do students learn and retain new information?}
\item
  \emph{How do students solve problems?}
\item
  \emph{How does learning transfer to new situations in or outside of the classroom?}
\item
  \emph{What motivates students to learn?}
\item
  \emph{What are common misconceptions about how students think and learn?}
\end{itemize}

{[}/ui-accordion-item{]}

{[}ui-accordion-item title=``Topic 2''{]}
\textbf{\emph{What is Coaching?}} Coaching is something that you do to help improve someone in some way. Gallway (1997) defines coaching as ``a way of being, listening, asking, and speaking that draws out and augments characteristics and potential that are already present in a person.'' His conceptualization of the method is analogous to Michelangelo's approach to sculpture. Michelangelo believed that ``every block of stone has a statue inside it and it's the task of the sculptor to discover it.'' Similarly, Gallway (1997) writes:

! \emph{Coaches know that an oak tree already exists within an acorn. They have seen the one grow into the other, over time and under the right conditions, and are committed to providing those conditions to the best of their abilities. Successful coaches continually learn how best to ``farm'' the potential they are given to nurture.}

\textbf{\emph{How do Educators Coach Learners for Learning?}} The beginning point of an effective coaching for learning practice as an educator is the coaching relationship. It's critical, according to Gallway (1997), for coaches to ``{[}create{]} a safe and challenging environment in which learning can take place.'' In the educator's role as coach, this relationship is a shared learning space where the educator enters the learners' internal dialogue with their learning experiences. Coaching work involves two aims, \textbf{(a)} helping each learner become aware of their own potential and \textbf{(b)} helping learners remove any interference in realizing this potential. In terms of coaching for learning we know that it is a natural aspect of our nature as human beings to learn. That is, it's something we simply do naturally. But, one thing that makes learning difficult is that our past learning gradually starts to interfere with our present and future learning.

Increasing learners' awareness of their own internal resistance to new learning is the primary theme of coaching for learning. Common obstacles to learning, according to Gallway (1997), include the learner's assumption that they already know what is being taught, the fear of being judged, doubt, and trying too hard to appear learned. In broader terms, Scharmer (2016) has identified the learners' internal \textbf{\emph{voice of judgement}} (VoJ), \textbf{\emph{voice of cynicism}} (VoC), and \textbf{\emph{voice of fear}} (VoF) as universal factors interfering with learning. The role of effective coaching is to help learners overcome these obstacles by opening their mind, emotions, and will to new possibilities. This work involves coaching learners through the transition of change---that is, \textbf{(a)} letting go of what was, \textbf{(b)} moving through the transition, and \textbf{(c)} embracing a new beginning.

The secondary theme of coaching for learning is increasing learners' awareness of how they may best realize their potential capacity to learn anything which they set out to learn. This involves coaching learners in the learning process and helping them make it more effective and efficient. This practice builds on the notion of transforming the learners' natural learning process into a disciplined set of study skills. A helpful approach that educators can use to coach for learning potential is the GROW coaching model. This model is based on the coaching concepts pioneered by Gallway (1997) and developed at the McKinsey consulting firm in the 1980s and first published in 1992 by John Whitmore in his book \emph{Coaching for Performance}. The model is widely used in leadership and life coaching contexts. For educators concerned with coaching for learning, we can apply this model in the following way:

!!!! \textbf{Goal Setting.} Helping learners clarify \textbf{what} they need and more importantly want to learn? \textbf{Why} they need and/or truly want to learn it? And \textbf{when} they need to, or genuinely want, to learn it? The coach, here, also helps learners see how their personal goals can align with prescribed learning outcomes.

!!!! \textbf{Reality Checking.} Helping learners identify what they already know? What they can already do? What they have already done? What's moving them towards their goal? And what's getting in the way?

!!!! \textbf{Option Exploring.} Helping learners identify the various options they have to move them toward their goals. Brainstorming what else they can do to achieve their goals? Assessing what are the benefits and weaknesses of the various options that they have identified?

!!!! \textbf{Will be Doing.} Helping learners choose which options to act on. Determining when to act on an option. Assessing how committed they are to taking action a given option? Committing to acting on an option.

{[}/ui-accordion-item{]}

{[}ui-accordion-item title=``Learning Activity''{]}

\hypertarget{fa-book-read-and-reflect}{%
\subsubsection{:fa-book: Read and Reflect}\label{fa-book-read-and-reflect}}

Take a few moments to read the article below written by Tim Gallwey. Gallwey works with companies to help them find better ways to implement change. In the article below, he discusses the importance of creating a learning culture - specifically, he discusses why coaches need to understand the learning process and the obstacles a learner experiences:

\begin{itemize}
\tightlist
\item
  \href{https://thesystemsthinker.com/the-inner-game-of-work-building-capability-in-the-workplace/}{\textbf{The Inner Game of Work}}
\end{itemize}

\textbf{\emph{Questions to Consider\ldots{}}}

After completing the activity above, consider the following questions:

\begin{itemize}
\tightlist
\item
  \emph{How can coaching concepts be applied to helping learners learn?}
\item
  \emph{What key characteristics define effective coaching for learning?}
\item
  \emph{How can educators coach learners through the transition of making a change?}
\end{itemize}

{[}/ui-accordion-item{]}

{[}ui-accordion-item title=``Topic 3''{]}
Effective coaching is grounded in an emerging set of coaching competencies. These competencies represent a set of integrated knowledge, skills, aptitudes and attributes that coalesce into behaviors that define, in more detail, what is needed to successfully perform the task of helping learners learn. The following are six essential behaviors the developing educator as coach must demonstrate.

\textbf{Practicing Professional Ethics \& Standards}

The basis of this competency is a personal commitment to demonstrating and maintaining the highest level of ethical behavior. Emerging professional coaching standards include,\\
:fa-check: making the roles, responsibilities, and rights of everyone involved in the coaching relationship clear,\\
:fa-check: clearly communicating how information will be communicated between everyone involved in the coaching relationship,\\
:fa-check: maintain confidentiality of personal information and communications,\\
:fa-check: having a clear understanding of the conditions where information will not be kept confidential, and\\
:fa-check: be aware of and respond sensitively to potential power or status differences.

\begin{center}\rule{0.5\linewidth}{0.5pt}\end{center}

\hypertarget{cultivating-trust-safety}{%
\subsubsection{Cultivating Trust \& Safety}\label{cultivating-trust-safety}}

Seeks to understand the learner and their unique learning context, demonstrates respect for the learner's identity, values and perspective, showing empathy and concern for their well-being, acknowledges the learner unique abilities, interests, and feelings, demonstrating openness and vulnerability, and supporting the unique needs of the learner's personality.

\begin{center}\rule{0.5\linewidth}{0.5pt}\end{center}

\hypertarget{holding-spacepresence}{%
\subsubsection{Holding Space/Presence}\label{holding-spacepresence}}

Being focused on the moment and giving attention to what is occurring in your conversation with the learner. Initially, this might feel like simply holding yourself back from talking too much as the coach and giving the learner space (that is, your silence) to reflect and process things. One way to conceptualize this competency as ``creating a ``space to listen into.'' This involves\\
:fa-check: creating comfortable distance in physical space,\\
:fa-check: creating a non-judgmental emotional space, and\\
:fa-check: creating a quiet auditory space. Holding space, or being present, is an essential way the educator as coach can demonstrate to the learner that the learner's contribution is valued and their learning is important. It's also important for the coach to hold space for themselves to reflect and process what is happening.

\begin{longtable}[]{@{}
  >{\raggedright\arraybackslash}p{(\columnwidth - 0\tabcolsep) * \real{0.0556}}@{}}
\toprule
\begin{minipage}[b]{\linewidth}\raggedright
\#\#\#\# Active Listening
\end{minipage} \\
\midrule
\endhead
\#\#\#\# Evoking Awareness \\
Asks questions to clarify the learners' experiences, way of thinking, feelings, values, needs, wants, and beliefs. Asking questions to help learners explore beyond their current thinking. Invites learners to share more about their present learning experiences. Notices what is working to enhance the learners progress. Challenges learners to increase their awareness and insight. Helps learners identify their potential for growth, internal/external obstacles, and options to move forward. Shares observations, insights, and encouragements to assist the learner in their learning. \\
\bottomrule
\end{longtable}

\hypertarget{cultivating-growth}{%
\subsubsection{Cultivating Growth}\label{cultivating-growth}}

Asks questions that helps learners\\
:fa-check: surface their interests,\\
:fa-check: clarify and prioritize their goals,\\
:fa-check: assess their learning experiences,\\
:fa-check: explore options for continued learning, and\\
:fa-check: design learning plans and commit to achieving their learning goals.

{[}/ui-accordion-item{]}

{[}ui-accordion-item title=``Learning Activity''{]}
\#\#\#\# :fa-book: Read and Reflect

For this activity, you will be reading through the Core Competencies of the International Coaching Federation. The International Coaching Federation is dedicated to advancing the coaching profession by setting high, ethical standards. Read more below:

\begin{itemize}
\tightlist
\item
  \href{https://coachfederation.org/core-competencies}{\textbf{Core Competencies}}
\end{itemize}

\textbf{\emph{Questions to Consider\ldots{}}}

After completing the activity above, consider the following questions:

\begin{itemize}
\tightlist
\item
  \emph{What general coaching competencies apply to coaching for learning?}
\item
  \emph{What competencies do you feel are strength areas? What areas do you need to develop?}
\item
  \emph{How can a coaching approach help you support learner success in an educational setting?}
\end{itemize}

{[}/ui-accordion-item{]}

{[}/ui-accordion{]}

\hypertarget{unit-4-assessment}{%
\subsection{Unit 4 Assessment}\label{unit-4-assessment}}

Please see the details for Post 4

\href{../assignments/_blog}{plugin:content-inject}

In addition, please begin planning your Peer Coaching Session, as below.

\href{../assignments/_peer-coaching}{plugin:content-inject}

  \bibliography{book.bib}

\end{document}
